%% ----------------------------------------------------------------
%% Thesis.tex -- MAIN FILE (the one that you compile with LaTeX)
%% ---------------------------------------------------------------- 

% Set up the document
\documentclass[a4paper, 11pt, oneside]{Relatorio_sem}  % Use the "Relatorio" style, based on the ECS Thesis style by Steve Gunn
\graphicspath{{Figure/}}  % Location of the graphics files (set up for graphics to be in PDF format)


% Include any extra LaTeX packages required
\usepackage[utf8]{inputenc}
\usepackage[brazilian]{babel}
\usepackage[T1]{fontenc}
\usepackage{ae,aecompl}
\usepackage[square, numbers, comma, sort&compress]{natbib}  % Use the "Natbib" style for the references in the Bibliography
\usepackage{verbatim}  % Needed for the "comment" environment to make LaTeX comments
%\usepackage{vector}  % Allows "\bvec{}" and "\buvec{}" for "blackboard" style bold vectors in maths
%\usepackage{listings}
\usepackage{listingsutf8}
\lstset{frame=tb,
  language=Python,
  aboveskip=3mm,
  belowskip=3mm,
  showstringspaces=false,
  columns=flexible,
  basicstyle={\small\ttfamily},
  numbers=none,
  numberstyle=\tiny\color{gray},
  keywordstyle=\color{dkblue},
  commentstyle=\color{dkgreen},
  stringstyle=\color{mauve},
  breaklines=true,
  breakatwhitespace=true,
  tabsize=4,
  lineskip={-1.5pt},
  morekeywords={models, lambda, forms, =}
  inputencoding=utf8,
  extendedchars=\true
}

\lstnewenvironment{code}[1][]%
  {\minipage{\linewidth} 
   \lstset{basicstyle=\ttfamily\footnotesize,frame=tb,#1}}
  {\endminipage}


%%%% Color definitions
\usepackage{color}
\definecolor{dkgreen}{rgb}{0,0.6,0}
\definecolor{dkblue}{rgb}{0,0,0.5}
\definecolor{gray}{rgb}{0.5,0.5,0.5}
\definecolor{mauve}{rgb}{0.6,0,0.82}
\definecolor{coolblack}{rgb}{0.0, 0.18, 0.39}
\hypersetup{urlcolor=coolblack, colorlinks=true}  % Colours hyperlinks in blue, but this can be distracting if there are many links.
%\hypersetup{urlcolor=black, colorlinks=false}  

%% To show a border around all figures
%\usepackage{float}
%\floatstyle{boxed} 
%\restylefloat{figure}

%% ----------------------------------------------------------------
\begin{document}
\frontmatter	  % Begin Roman style (i, ii, iii, iv...) page numbering

% Set up the Title Page
\title  {Baobáxia}

\authors  {\texorpdfstring
            {\href{mailto:vince@mocambos.net}{Vincenzo Tozzi}} - 
            {\href{mailto:fernao@riseup.net}{Fernão Lopes}}
            }
\addresses  {\taina\\\npdd\\\redemocambos}  % Do not change this here, instead these must be set in the "Thesis.cls" file, please look through it instead
\date       {\today}
\subject    {Relatório de execução do projeto}
\keywords   {Federated Network, Tecnological Autonomy, Free Software, Rede Mocambos, Casa de Cultura Taina, Quilombo}

\maketitle
%% ----------------------------------------------------------------

\setstretch{1.3}  % It is better to have smaller font and larger line spacing than the other way round

% Define the page headers using the FancyHdr package and set up for one-sided printing
\fancyhead{}  % Clears all page headers and footers
\rhead{\thepage}  % Sets the right side header to show the page number
\lhead{}  % Clears the left side page header

\pagestyle{fancy}  % Finally, use the "fancy" page style to implement the FancyHdr headers

% %% ----------------------------------------------------------------
% % Declaration Page required for the Thesis, your institution may give you a different text to place here
% \Declaration{

% \addtocontents{toc}{\vspace{1em}}  % Add a gap in the Contents, for aesthetics

% Io, Vincenzo Tozzi, dichiaro che la presente tesi, "Reti federate eventualmente connesse" e il lavoro presentato in essa sono di mia paternità. Dichiaro che:

% \begin{itemize} 
% \item[\tiny{$\blacksquare$}] Questo lavoro è stato totalmente o per la maggior parte svolto come laureando per la Laurea in Informatica di questa Università.
 
% \item[\tiny{$\blacksquare$}] Ho esplicitamente dichiarato nel testo se qualche parte di questa tesi è stata precedentemente pubblicata in altri lavori da questa o altre Università o istituzioni.
 
% \item[\tiny{$\blacksquare$}] Ho attribuito la paternità ai lavori pubblicati da altri e da me consultati.
 
% \item[\tiny{$\blacksquare$}] Ho sempre citato la fonte di opere altrui. Ad eccezione di tali citazioni, questa tesi è di mia paternità.
 
% \item[\tiny{$\blacksquare$}] Ho ringranziato tutte le principali fonti di supporto.
 
% \item[\tiny{$\blacksquare$}] Ho esplicitamente dichiarato del testo, in parti sviluppate assieme ad altri, qual'è il loro e il mio contributo.
% \\
% \end{itemize}
 
 
% Firmato:\\
% \rule[1em]{25em}{0.5pt}  % This prints a line for the signature
 
% Data:\\
% \rule[1em]{25em}{0.5pt}  % This prints a line to write the date
% }
% \clearpage  % Declaration ended, now start a new page

%% ----------------------------------------------------------------
% The "Funny Quote Page"
\pagestyle{empty}  % No headers or footers for the following pages

\null\vfill
% Now comes the "Funny Quote", written in italics
\textit{``Vamos fazer um mundo mais do nosso jeito...''}

\begin{flushright}
Zumbi dos Palmares
\end{flushright}

\vfill\vfill\vfill\vfill\vfill\vfill\null
\clearpage  % Funny Quote page ended, start a new page
%% ----------------------------------------------------------------

% The Abstract Page
\addtotoc{Resumo}  % Add the "Abstract" page entry to the Contents
\abstract{

\addtocontents{toc}{\vspace{1em}}  % Add a gap in the Contents, for aesthetics

O projeto Baobáxia trata da concepção, desenvolvimento e implementação
de uma arquitetura distribuída, voltada para a integração de redes
locais mesmo em localidades nas quais a conexão à internet seja
instável, lenta ou intermitente. Parte-se da experiência acumulada
pela Rede Mocambos, que trabalha com a integração de mais de duzentas
comunidades em todas as regiões do país através da apropriação de
tecnologias, identificando pontos críticos em que a precariedade do
acesso à internet se torna um impeditivo para a efetiva comunicação
entre essas comunidades. O projeto parte do pressuposto de que não
basta usar as tecnologias de informação já existentes - é preciso
moldar o próprio desenvolvimento dessas tecnologias, para que atendam
às demandas da sociedade. Para isso, adota como princípio básico e
metodologia de trabalho os fundamentos do software livre - tanto na 
gestão das equipes de trabalho quanto nas soluções tecnológicas que 
utilizará.
}

\clearpage  % Abstract ended, start a new page
%% ----------------------------------------------------------------

\setstretch{1.3}  % Reset the line-spacing to 1.3 for body text (if it has changed)

% The Acknowledgements page, for thanking everyone
%% \acknowledgements{
%% \addtocontents{toc}{\vspace{1em}}  % Add a gap in the Contents, for aesthetics

%% Um agradecimento especial para as nossas familias\ldots

%% }
\clearpage  % End of the Acknowledgements
%% ----------------------------------------------------------------

\pagestyle{fancy}  %The page style headers have been "empty" all this time, now use the "fancy" headers as defined before to bring them back


%% ----------------------------------------------------------------
\lhead{\emph{Contents}}  % Set the left side page header to "Contents"
\tableofcontents  % Write out the Table of Contents

%% ----------------------------------------------------------------
%\lhead{\emph{List of Figures}}  % Set the left side page header to "List if Figures"
%\listoffigures  % Write out the List of Figures

%% ----------------------------------------------------------------
%\lhead{\emph{List of Tables}}  % Set the left side page header to "List of Tables"
%\listoftables  % Write out the List of Tables

%% ----------------------------------------------------------------
\setstretch{1.5}  % Set the line spacing to 1.5, this makes the following tables easier to read
\clearpage  % Start a new page
\lhead{\emph{Acrónimos}}  % Set the left side page header to "Abbreviations"
\listofsymbols{ll}  % Include a list of Abbreviations (a table of two columns)
{
% \textbf{Acronym} & \textbf{W}hat (it) \textbf{S}tands \textbf{F}or \\
  \textbf{RM} & \textbf{R}ede \textbf{M}ocambos\\ \textbf{SP} &
  \textbf{S}ervice \textbf{P}rovider\\ \textbf{IdP} &
  \textbf{Id}entity \textbf{P}rovider\\ \textbf{API} &
  \textbf{A}pplication \textbf{P}rogramming
  \textbf{I}nterface\\ \textbf{RFC} & \textbf{R}equest \textbf{F}or
  \textbf{C}omments\\ \textbf{JSON} & \textbf{J}ava\textbf{S}cript
  \textbf{O}bject \textbf{N}otation\\ \textbf{P2P} & \textbf{P}eer To
  \textbf{P}eer\\ \textbf{LDAP} & \textbf{L}ightweight
  \textbf{D}irectory \textbf{A}ccess \textbf{P}rotocol\\ \textbf{YAML}
  & \textbf{Y}et \textbf{A}nother \textbf{M}arkup
  \textbf{L}anguage\\ \textbf{XMPP} & e\textbf{X}tensible
  \textbf{M}essaging and \textbf{P}resence
  \textbf{P}rotocol\\ \textbf{SSO} & \textbf{S}ingle \textbf{S}ign
  \textbf{O}n\\ \textbf{VSAT} & \textbf{V}ery \textbf{S}mall
  \textbf{A}perture \textbf{T}erminal\\ \textbf{DRY} & \textbf{D}on't
  \textbf{R}epeat \textbf{Y}ourself\\ \textbf{MVC} & \textbf{M}odel
  \textbf{V}iew \textbf{C}ontroller\\ \textbf{ORM} & \textbf{O}bject
  \textbf{R}elational \textbf{M}apper\\ \textbf{NPDD} &
  \textbf{N}úcleo de \textbf{P}esquisa e \textbf{D}esenvolvimento
  \textbf{D}igital\\ \textbf{NFC} & \textbf{N}úcleo de
  \textbf{F}ormação \textbf{C}ontinuada\\ \textbf{NCP} &
  \textbf{N}úcleo de \textbf{C}omunicação e
  \textbf{P}edagogia\\ \textbf{GESAC} & \textbf{G}overno
  \textbf{E}letrônico \textbf{S}erviço de \textbf{A}tendimento ao
  \textbf{C}idadão\\ }

%% ----------------------------------------------------------------
%\clearpage  % Start a new page
%\lhead{\emph{Physical Constants}}  % Set the left side page header to "Physical Constants"
%\listofconstants{lrcl}  % Include a list of Physical Constants (a four column table)
%{
%% Constant Name & Symbol & = & Constant Value (with units) \\
%Speed of Light & $c$ & $=$ & $2.997\ 924\ 58\times10^{8}\ \mbox{ms}^{-\mbox{s}}$ (exact)\\
%
%}

%% ----------------------------------------------------------------
%\clearpage  %Start a new page
%\lhead{\emph{Simboli}}  % Set the left side page header to "Symbols"
%\listofnomenclature{lll}  % Include a list of Symbols (a three column table)
%{
%% symbol & name & unit \\
%$a$ & distance & m \\
%$P$ & power & W (Js$^{-1}$) \\
%& & \\ % Gap to separate the Roman symbols from the Greek
%$\omega$ & angular frequency & rads$^{-1}$ \\
%}
%% ----------------------------------------------------------------
% End of the pre-able, contents and lists of things
% Begin the Dedication page

%% \setstretch{1.3}  % Return the line spacing back to 1.3

%% \pagestyle{empty}  % Page style needs to be empty for this page
%% \dedicatory{Para nossa Mãe\ldots}

%% \addtocontents{toc}{\vspace{2em}}  % Add a gap in the Contents, for aesthetics


%% ----------------------------------------------------------------
\mainmatter	  % Begin normal, numeric (1,2,3...) page numbering
\pagestyle{fancy}  % Return the page headers back to the "fancy" style

% Include the chapters of the thesis, as separate files
% Just uncomment the lines as you write the chapters

\part{Julho/Agosto}

\chapter{Organização e inicio do projeto}\lhead{\leftmark}
O projeto inicia formalmente com as assinaturas no dia 17 de junho na
agencia BB de Campinas. O NPDD responsável pela condução do projeto
realiza algumas reuniões para definir o grupo de trabalho da Rede
Mocambos que vai ser contratada e as metas e etapas do projeto.

\section{Encontro de Culturas Tradicionais/GO}
A Rede Mocambos participou do XIII Encontro de Culturas Tradicionais
da Chapada dos Veadeiros, e pela ocasião organizamos uma reunião do
projeto no Mercado Sul de Taguatinga.

\begin{figure}[htbp]
  \centering
  \includegraphics[width=\textwidth]{./Fig/Rota_Kalungas_1.pdf}
  \rule{35em}{0.5pt}
  \caption[Encontro quilombola na Chapada dos Veadeiros]{Encontro quilombola na Chapada dos Veadeiros}
  \label{fig:EncontroKalungas}
\end{figure}

\section{Oficinas de introdução a instrumentalização tecnológica/MA}
Por Milson Onileto, em 26 julho
2013\footnote{\url{http://culturadigital.br/casaferreirodedeus/2013/07/26/ola-mundo/}}:
\begin{quote}
  Rede Mocambos maranhão/Centro Cultural Alagbedê, vem realizando
  atividades de formações continuada em parceria com a Escola de
  Capoeira Angola Mandingueiros do Amanha.  As oficinas focadas em
  apropriação tecnológica em software livre linux e ferramentas
  mocambolicas, trabalhando o social através do digital. entendendo
  que nossos territórios transcende o geográfico. as aulas acontecem
  todos os sábado na sede da escola na rua Portugal, a parti das 09:00
  da manha.  As atividades são divididas como aulas para os alunos do
  projeto e um horário especial para os pais e mães dos mesmos.
  
  As formações são as primeiras de uma serie itinerante de formações
  pelos núcleos da rede no estado e parceiros.
\end{quote}

\begin{figure}[htbp]
  \centering
  \includegraphics[width=\textwidth]{./Fig/Oficina_julho_MA_milson_1.pdf}
  \rule{35em}{0.5pt}
  \caption[Oficinas no Centro Cultural Alagbedê]{Oficinas no Centro Cultural Alagbedê}
  \label{fig:OficinaMA1}
\end{figure}
\begin{figure}[htbp]
  \centering
  \includegraphics[width=\textwidth]{./Fig/Oficina_julho_MA_milson_2.pdf}
  \rule{35em}{0.5pt}
  \caption[Oficinas no Centro Cultural Alagbedê]{Oficinas no Centro Cultural Alagbedê}
  \label{fig:OficinaMA2}
\end{figure}

Por Milson Onileto, em 25 agosto
2013\footnote{\url{http://culturadigital.br/casaferreirodedeus/2013/08/25/segunda-oficina-de-apropriacao-tecnologica/}}:
\begin{quote}
  Nessa manha do dia 24.08.2013 aconteceu a segunda oficina de
  apropriação tecnológica,a oficina partiu da ideia que temos que
  brigar por um territorio tecnologico que vem nos sendo negado ou vem
  sendo usado de forma não-critica, pela falta de acesso a informações
  e instrumentos tecnológicos livre. A rede mocambos vem criando esse
  vieis entre as tradicionais e o mundo tecnológico moudado ao nosso
  modo criando um mundo cada vez mais do nosso jeito.
\end{quote}

\section{Debian Day/ES}
No 17 de Agosto, o Núcleo de Cidadania Digital (NCD), juntamente com o
Grupo de Usuários Debian do Espírito Santo (Gud-ES), realizou, no
Centro de Artes da Ufes, o Debian Day, que contou com a participação
da Rede Mocambos falando sobre apropriação tecnológica.  O Debian Day
trata-se de uma comemoração aberta do aniversário do Projeto Debian,
cuja data de lançamento oficial foi 16 agosto de 1993, visa a produção
de um sistema operacional livre.  Dentre das atividades, ocorridas uma
foi a apresentação da rede mocambos falando sobre apropriação
tecnológica das comunidades quilombolas e movimentos sociais, apos a
apresentação rolou uma conversar com os participantes e organização do
evento, surgindo possibilidades de parcerias com o Núcleo de Cidadania
Digital (NCD), Grupo de Usuários Debian (Gud-ES), Tux-ES e Revista
Espirito
Livre.\footnote{\url{http://elegbaraguine.wordpress.com/2013/08/24/debian-day-vitoria-es/},
  \url{http://softwarelivre.org/gud-es},
  \url{http://www.mocambos.net/weblog/2013/08/24/rede-mocamdos-no-debian-day-vitoria-es/}.}

\chapter{NPDD}\label{NPDD}\lhead{\leftmark}
O Núcleo de Pesquisa e Desenvolvimento Digital, NPDD, envolve pessoas
com conhecimento técnicos de varias comunidade e realidades da Rede. O
NPDD é responsável pelo desenvolvimento e manutenção das ferramentas
digitais da Rede, cuidando atualmente dos portais (www.mocambos.net,
wiki.mocambos.net, mapa.mocambos.net, galeria.mocambos.net), das
contas email (@mocambos.net e @mocambos.org) e de criar documentação
de base sobre as ferramentas digitais.

\section{Informações}

\subsection{Endereços e contatos}
Para informações e contatos podem escrever ao seguinte email, nosso
principal canal de comunicação: \\ \url{suporte@mocambos.org}

Existe também uma lista de discussão do NPDD hospedada no riseup.net:
\\ \url{mocambos-npdd@lists.riseup.net}

\begin{tabular}{lll}

\parbox[t]{0.3\textwidth}{
        \textbf{Distrito Federal} \\
        Mercado Sul \\
        QSB 12/13 Loja 7 \\
        Taguatinga/DF \\
}
        &
\parbox[t]{0.3\textwidth}{
        \textbf{São Paulo} \\
        Casa de Cultura Tainã \\
        Rua Inhambú, 645 \\
        Campinas/SP \\
        Telefone: (19) 32282993 \\
}
        &
\parbox[t]{0.35\textwidth}{
        \textbf{Sicilia/Itália} \\
        BOCS \\
        Via Piersanti Mattarella, 8 \\
        Bagheria \\
}

\end{tabular}

\section{Consultorias}
\subsection{Dynamite}
A versão 0.1 do projeto que foca na questão de metadados, conforme
plano de trabalho aprovado, foi desenvolvida com a consultoria da
Associação Cultural Dynamite (nota fiscal 0000999).


\section{Metodologia}
O projeto é conduzido pelo NPDD (Núcleo de Pesquisa e Desenvolvimento
Digital da Rede Mocambos), em sinergia com os demais Núcleos (NCP de
Comunicação e Pedagogia, NFC de Formação Continuada). À equipe fixa de
desenvolvedores se soma a participação de colaboradores e
especialistas em diferentes áreas. Sob a coordenação do NPDD, o
desenvolvimento se da de forma distribuída, utilizando-se ferramentas
colaborativas via internet como git, wiki e irc. A metodologia de
desenvolvimento segue o modelo ``Agile'' que prevê o lançamento
frequente de código funcionante para avaliação contínua por parte dos
usuários finais, permitindo a correção e a melhoria ao longo do
trabalho.

\subsection{Wiki}
A documentação do projeto é disponível no wiki da Rede Mocambos no
endereço: \\ \url{http://wiki.mocambos.net/NPDD/Baobáxia}

\subsection{Codigo}
O codigo do projeto é disponível em licença GPLv3 no Github no
endereço: \\ \url{http://github.com/RedeMocambos/baobaxia}

\subsection{Necessidades/Issues}
As necessidades do projeto são registradas no sistema de issues do
github no endereço:
\\ \url{http://github.com/RedeMocambos/baobaxia/issues}

 % NPDD


\part{Setembro/Outubro}
\chapter{Estruturando as Rotas locais e os Metadados}
O projeto continua com as articulações e oficinas locais no Rio Grande
do Sul e no norte. A equipe de desenvolvimento, com o apoio de
consultores da Dynamite, realiza um estudo e uma primeira
implementação da gestão dos metadados do Baobáxia.

\section{Articulando com os terreiros}
No dia 11 de setembro de 2013, aconteceu uma reunião com
representantes do Mansu Nangetu terreiro da Mãe Nangetu, onde a
possibilidade de fazer uma imersão sobre a rede mocambos com as
comunidades de terreiro de Belém do Pará foi fechada. A data pra a
oficina de instrumentalização tecnológica e ferramentas mocambólicas
com datas de 08 à 10 do mês de Outubro (data sujeita à alterações),
tendo em vista uma utilização critica da internet que de acordo com
dados de pesquisas nacionais relaciona a porcentagem de pouca
utilização dessa ferramenta por parte das comunidades negras e
tradicionais de terreiro.

Durante a conversa, foi colocado por Milson Onilètó a importância da
apropriação desse território tecnológico, e dessa forma fortalecer as
praticas tradicionais e das organizações politicas das comunidades.

Mãe Nangetu falou da articulação das comunidades tradicionais nas
discussões nacionais sobre as políticas públicas, principalmente as
culturais. Ela relatou que estão acontecendo constantemente as
conferências em Belém e que é importante que a juventude de terreiro
se aproprie destas discussões politicas e que uma organização maior é
necessária, que é responsabilidade da juventude manter essa luta ao
lado dos mais velhos, afinal, o conhecimento ancestral deve ser
repassado para preservarmos nossa história, nossas raízes. Arthur
Leandro falou um pouco de seu histórico dentro do movimento afro
religioso e de sua própria história de vida (que renderia um belo
livro) e se mostrou bastante interessado em compor essa articulação
local junto à rede mocambos, e que a casa está disposta a mobilizar na
data estabelecida as lideranças estratégicas para a juventude de
terreiro e dessa forma fortalecer mais um núcleo rede mocambos em
Belém do Pará.  

\section{Oficinas}
A oficina de apropriação tecnológica aconteceu no dia 9 de outubro de
forma natural, durante a oficina de vídeo, quando o grupo percebeu,
que era mais viável se instrumentalizar sobre as questões básicas de
software livre e informática. Dessa forma foi necessário readequá a
programação uma vez que seria apresentado a oficina de vídeo. A
oficina foi pensada em três momentos de atividade que eram introdução
a história dos computadores e internet que é o momento para
procurarmos causar uma reflexão crítica desses da história e criação
desses meios, aproveitando para falar de Rede Mocambos seu ideal de
luta e de como nos inspiramos na causa palmarina para discutir esse
território digital excludente. Foi mostrado filmes curtos depois de
apresentação de slides sobre o tema. Esse filmes falaram sobre a
funcionalidade das principais peças na maquina. Dessa forma esse vídeo
foi um complemento que de forma visual facilitou o entendimento do que
é o computador.

O segundo momento foi pensado para contextualização desse entendimento
tecnológico e a pratica propriamente dita. Partindo pela instalação do
software, utilização dos principais comandos e instalação dos
principais programas. Esse momento foi acontecendo a parti de
provocações do grupo.

O terceiro momento foi pensado para falar sobre as ferramentas
tecnológicas da Rede Mocambos, onde falamos da historia da wiki e sua
forma de utilização. Passamos pelo processo de criação de usuário e
todos os passos para criação dos publicação de conteúdos. O mapa foi
apresentado e foi descrito sua importância para as comunidades porem o
tempo não foi o bastante para mostra sua utilização.

\section{Oficina em Bujarú/PA}
A atividade na comunidade Quilombola de São judas Tadeu foi
linda. Chegamos Um hora da tarde quando o Aldo representante
quilombola foi nos pegar na balsa e levar pra comunidade, a presidente
da associação Kátia recebeu muito bem a equipe, relatando a situação
da comunidade e que eles já esparavão a anos essa instalação. Foi
notório a euforia da comunidade que logo se aproximou e quiz saber se
seu sonho se realizaria naquele momento. O que deveria começar no
outro dia pela manha, iniciou-se no mesmo passo da chegada. Começamos
a instalação do telecentro e com as oficinas que iam acontecendo
naturalmente como conversas, os jovens e crianças, mulheres e homens
da comunidade todos interagindo com o processo. Logo cada um estava
contando pela primeira vez um computador. depois da instalação e da
troca de ideias sobre informática e apropriação tecnológica, começamos
a falar da realidade das pessoas que viviam ali, e que muitos se quer
ja tinha assistido filme em uma tela grande. começamos a pensar em uma
mostra de cinema que aconteceu no espaço do telecentro, logo a sala
esta cheia de crianças, jovens, mulheres, todo mundo juntos, vidrados
nas aventuras de Kiriku, o filma acabou e observar os olhares pidões
querendo mais das crianças foi o bastante pra em poucos cliques,
começar o filme Quilombos, que conta a historia do Quilombo dos
Palmares. depois do filme as atividades terminarão. Pela manhã
começamos ouvindo os mais velhos da comunidade e visitando as áreas de
plantações que rodeava toda a comunidade com uma bela forma de
sustentabilidade, ouvimos como começarão as plantações e conhecemos um
dos igarapé que cortava e dividia as comunidades. Após todo o rolê
fomos para o telecentro onde terminamos a instalação e demos
continuidade as oficinas dessa vez de instalação e noções básicas de
Linux Ubuntu ( comandos e principais programas, entendendo as
principais diferenças entre o sistema operacional Windows e o Linux,
observamos que ninguém si quer tinha ouvido falar do Linux antes. Após
essa atividade fizemos a segunda mostra de cinema com curtas para
criançada com temáticas ambientais. Isso durante a instalação demorada
de todo o sistema do telecentro e configuração do Roteador wireless.

\chapter{Metadados}\label{Metadados}\lhead{\leftmark}

\section{Descrição geral da pesquisa de Metadados e políticas de metadados}
A pesquisa sobre metadados para o sistema Baobáxia compreendeu as
definições diretamente relacionadas aos metadados das mídias e também
a participação na modelagem geral dos objetos do sistema, assim como a
definição de formatos para troca de dados. Por metadados entendemos
toda a informação a respeito de si, seja para a mídia, seja para as
Mucuas como para o sistema. Deste modo, num entendimento amplo de
metadados, consideramos que qualquer dado descritivo a respeito do
sistema como um todo deve ser entendido como um metadado, e logo deve
ser armazenado em arquivos de definição.

Essa diretiva partiu de uma orientação em conjunto com o
desenvolvimento do núcleo do software Baobáxia
(https://github.com/RedeMocambos/baobaxia). Deste modo, tudo que
envolve a descrição de elementos do sistema pode ser considerado um
metadado. Os metadados, embora tenham como objetivo a estabilidade da
informação, devem contemplar a possibilidade de alterações nas suas
definições, bem como a expansão futura.

De início, definem-se dois tipos principais de metadado:
\begin{itemize}
\item metadado do sistema / políticas (arquivos de configuração,
  regras de funcionamento)
\item metadado do acervo/das mídias
\end{itemize}

Não coube à pesquisa em metadados a definição das políticas, mas a
orientação de como estas deveriam ser armazenadas e estruturadas. As
políticas de sistema dizem respeito a definições gerais sobre a
configuração do software e de critérios estabelecidos politicamente,
isto é, atendendo às demandas da comunidade usuária e gestora do
sistema Baobáxia. Entretanto, há uma escolha que entende a importância
da distinção entre o sistema e suas configurações, estas que devem ser
implementadas de acordo com as necessidades locais da Mucua ou do
repositório (baobáxia / Rede Mocambos).

A partir da definição de políticas, ocorre a tomada de decisões pelo
sistema, respeitando aos critérios estabelecidos. Essa definição
possui normas técnicas próprias, mas é abrangente o bastante para
permitir que novas políticas sejam incorporadas ao sistema conforme
surjam novas necessidades.

\section{Mídias e Mucuas}
Os metadados relacionados a mídias e mucuas constituem o núcleo do
sistema. Em tratando-se de um sistema de acervo distribuído, os nós
(Mucuas) e arquivos contidos nesses nós (Mídias) são os principais
elementos do sistema do ponto de vista dos conteúdos.

As mídias sempre estão associadas a mucuas originadoras (as quais originaram a publicação), e dividem-se por tipos:
\begin{itemize}
\item vídeos
\item imagens (fotografias, desenhos etc)
\item áudios (músicas, entrevistas, programas de rádio etc)
\item arquivos (documentos, textos e demais arquivos)
\end{itemize}

Cada tipo de mídia possui especificidades do ponto de vista de seus
metadados, ao que buscou-se definir o que há de comum entre todas. Os
metadados foram definidos preliminarmente como o mínimo essencial
possível, sendo que qualificadores adicionais devem ser estendidos ou
como tags ou como metadados específicos do tipo de arquivo. Assim,
chegou-se aos seguintes campos, cuja definição encontra-se nos
arquivos de políticas (ver adiante):
\begin{itemize}
\item data (AAAA-MM-DD)
\item uuid (identificação única do arquivo)
\item title (título, texto não único)
\item comment (comentário, texto aberto)
\item author (mocambola, associado a quem publicou o arquivo)
\item type (tipos das mídias [vídeo|imagens|áudios|arquivos])
\item format (formato do arquivo, definido em arquivo de políticas)
\item origin (mucua, o servidor a partir do qual arquivo foi
  publicado)
\item repository (referência ao repositório git annex a que está vinculado o arquivo)
\item tags (etiquetas, múltiplas; somente descritoras ou também
  funcionais)
\end{itemize}

Além da mídia, as mucuas (nós do acervo, servidores locais) possuem
também metadados específicos, bem como associados. Os metadados
específicos dizem respeito a informações ligadas diretamente ao
servidor; as associadas, ao conteúdo hospedado no servidor. Dessa
forma, temos:
\begin{itemize}
\item note (texto geral sobre a mucua)
\item description (nome da mucua)
\item uuid (identificador único da mucua)
\end{itemize}


Quanto ao metadado associado, diz respeito a todas as etiquetas de
conteúdos que estejam hospedados na mucua. Está previsto como um
metadado descritivo sobre a mucua um relatório agrupando todas as
etiquetas dos arquivos hospedados, o que permite aos mocambolas
(usuári@s) ter uma ideia geral sobre as características do acervo de
determinada Mucua.

\section{Armazenamento do metadado: arquivos, padrões adotados e intercâmbio}
Conforma já assinalado, metadados e políticas são armazenados em
arquivos descritores. Para tal foi feita uma pesquisa a respeito dos
formatos a serem adotados para armazenamento destes dados. Teriam que
atender os seguintes requisitos:
\begin{description}
\item[A] armazenamento em suporte aberto, não proprietário
\item[B] ótima capacidade de intercâmbio entre linguagens
\item[C] amplo desenvolvimento de bibliotecas de código em distintas linguagens de programação
\item[D] capacidade para armazenamento de objetos complexos
\item[E] bom desempenho
\item[F] tamanho diminuto
\item[G] adequação a plataformas RESTful e serviços de dados (web services)
\end{description}

Inicialmente, levantou-se três possibilidades, todas elas atendendo os itens A e B:
\begin{description}
\item[XML] eXpansible Markup Language
\item[YAML] Yet Another Markup Languge
\item[JSON] JavaScript Object Notation
\end{description}

O primeiro formato, XML, tem a seu favor o fato de ser largamente
adotado para intercâmbio de dados e para descrição de informações já
há muitos anos (C). Conta no entanto com algumas desvantagens
sobretudo do ponto de vista do desempenho (E), além de gerar uma não
desprezível quantidade de informações extra especialmente ao lidar com
o armazenamento de objetos complexos, o que tem impactos severos ao se
pensar numa rede de acervos com conectividade baixa ou nula (ponto
F). Além disso, conta com possíveis problemas no armazenamento de
objetos complexos ao redundar numa estrutura de dados pesada (D).

O segundo formato, YAML, apresenta-se como um possível sucessor do
XML, tendo se inspirado neste. Tem como resultado arquivos sintéticos
e de tamanho diminuto (E e F). Possui ótima capacidade de
armazenamento de objetos complexos (D). Apesar de contar com mais
funcionalidades que o JSON - como a possibilidade de estabelecer
relacionamentos funcionais lógicos internos, conta ainda com menor
desenvolvimento e compatibilidade (C e G), ainda que seu futuro seja
promissor.

O terceiro formato, JSON, é considerado uma forma de reduzir o
overhead computacional do XML (E), sendo uma excelente alternativa
para armazenamento de objetos complexos (D), sendo sintético e de
tamanho diminuto (F). Conta com grande desenvolvimento em uma série de
softwares (C) pode-se dizer que atualmente é o novo padrão para
intercâmbio de dados, sendo base para numerosas tecnologias baseadas
em plataformas RESTful (G)

Desse modo, foi escolhido o formato JSON como padrão para intercâmbio
de dados, definição de políticas e metadados.

\section{Politicas}

\subsection{Media}

Definição dos metadados de mídia:
\begin{description}
  \item[formats] definem-se os tipos de formatos aceitos
  \item[priority] tipos prioritários aceitos pelo sistema
  \item[metadata] definição dos campos com tipo de dados aceito. Pode
    receber uma expressão regular (ex.: \verb| /^\w_-$/ |)
\end{description}

\lstinputlisting[basicstyle={\scriptsize\ttfamily}]{../../policies/media.json}

\subsection{Mucua}

\lstinputlisting[basicstyle={\scriptsize\ttfamily}]{../../policies/mucua.json}


 % Metadados



\part{Novembro/Dezembro}
\chapter{Infraestrutura e primeiras Mucuas}
O NPDD foi inaugurado com a chegada das 4 mucuas previstas no
projeto. A estruturação das Rotas locais continuou com a instalação de
alguns telecentros. O desenvolvimento do Baobáxia esta focado na
gestão de repositorios de diferentes redes.

\section{Mucuas}
Foram adquiridos 4 computadores como primeiras mucuas do
Baobáxia. Apos reunião em Brasilia em ocasião da inauguração da nova
sede do NPDD, se propus destinar as seguinte comunidades:
\begin{description}
  \item[dpadua] Mercado Sul de Taguatinga/DF
  \item[akoni] Comunidade Zumbi dos Palmares/MA 
  \item[exu] Terreiro Cultural Coco de Umbigada/PE
  \item[mestreborel] AfroSul Odomode/RS
\end{description}

\begin{figure}[htbp]
  \centering
  \includegraphics[width=\textwidth]{./Fig/Fotos_Mucuas.pdf}
  \rule{35em}{0.5pt}
  \caption[Fotos das Mucuas no NPDD]{Fotos das Mucuas no NPDD}
  \label{fig:Mucuas}
\end{figure}



\section{Instalação do telecentro de Pitimandeua/PA}

Dia 14 de Novembro de 2013, mais um telecentro quilombola foi
instalado e conectado.  A comunidade Quilombola Menino Jesus de
pitimandeua, localizada no munícipio de pitimandeua, proximo a
Castanhal-PA, recebeu os articuladores da rede mocambos Guinê Ribeiro
e DJ RG, que realizaram o processo de instalação do telecentro da
comunidade.  Essa iniciativa e uma ação conjunta dos projetos Baobaxia
e Núcleos de formação continuda da Rede
mocambos.\footnote{\url{http://www.mocambos.net/pt-br/weblog/2013/11/20/Destaque/}}


\section{Telecentro das Ilhas-Abaitetuba-Rio Genipauba/PA}

Dia 13 de novembro de 2013, a Rede mocambos instalou o telecentro da
associação de artesãs das ilhas de Abaitetuba, rio Genipauba.
http://galeria.mocambos.net/Instala-o-do-telecentro-das-Ilhas--Abaitetuba--Rio-Genipauba


\section{Visita aos quilombos de Guajará Mirim e Itacoã Mirim/PA}
Dia 12/11/2013 

Visita nas comunidades com o objetivo de instalar/verificar os
Telecentros, quais os materiais e maquinários entregues,
disponibilidade de conexão, estrutura e espaço como salas, escola,
associação, e a relação da comunidade com a tecnologia e
comunicação\footnote{\url{http://galeria.mocambos.net/Instala-o-dos-Telecentro-de-Guajar-e-Itaco--PA}}.
\begin{description}
  \item[Itacoã] Comunidade com aproximadamente 150 famílias, que
    desenvolvem atividades de plantio, produção de farinha, entre
    outras que fazem parte do cotidiano da região. Conversa sobre
    rotinas, informes locais (ano eleitoral, dezembro, para designar
    representantes legais da comunidade, propostas e melhorias),
    realizada na residência de seu Zeca (morador local), no Telecentro
    chegou apenas o data show, não tem conexão, mesas, computadores e
    espaço destinado à instalação.
  \item[Guajará Mirim] Comunidade tem em torno de 120 famílias,realizam
    atividades de manejo, plantio e produção, mas acabam por se
    limitar no processo de desenvolvimento do próprio espaço, através
    de projetos, parcerias e novos investimentos, decorrente dos
    interesses políticos e econômicos predominantes na região,
    empresários, gabinetes, prefeituras, tem retido o fluxo de
    investimentos predestinados à melhoria e crescimento das
    comunidades. Isso reflete na questão sócio econômica e ambiental
    da região, como o mal uso da terra (extração de areia) e suas
    propriedades, ocasionando o desequilíbrio ecológico,
    desobstrução/erosão do solo e a não produtividades e preservação
    da flora extensa e rica.  Telecentro na Escola de Ensino
    Fundamental Stª Marta, recebeu as mesas, 11 computadores, data
    show, mas não tem antena, conexão, a sala destinada para
    instalação não tem estrutura elétrica (fiação de modo geral). A
    conversa teve a participação e colaboração de Valdinei (Pedra),
    representante da comunidade e atuante do Projeto Ijé Ofé, Marco
    Antônio (cabeludo), vice-presidente da Associação.  A instalação
    dos telecentros não foi efetivada, pois em Itancoã não chegou
    maquinário, apenas data show e, em Guajará Miri o espaço reservado
    na escola não tem estrutura. Foram feitos registros de fotos e
    vídeos (Guine e Reginaldo-Dj.RG), entrevistas com os
    representantes sobre a realidade dos telecentros locais. As
    comunidades quilombolas reconhecem as suas necessidades,
    limitações, e mesmo com as problemáticas existentes a resistência
    e trabalho persistem, pois as ações tem continuidade, seja nas
    escolas, nas associações e com os próprios moradores. Os
    telecentros são importantes para a comunicação entre as
    comunidades, para o crescimento e projetos de ensino nas escolas,
    para a integração e envolvimento de jovens e adultos, no intuito
    de fortalecer os interesses (cultura, identidade, valores) e o
    comprometimento com a comunidade, atribuindo o que se tem de
    direito que vai muito além da terra.  Como encaminhamento, os
    representantes Pedra e Marcos, providenciarão um profissional
    eletricista para fazer um orçamento e ajustes necessários, no
    espaço que foi destinado para a instalação do Telecentro, à parti
    disso comunicar (agendar) a Rede para uma nova visita, contactar
    um correio mais próximo e ver as possibilidades de
    entrega/endereço, para receber antena ou qualquer outro maquinário
    que for preciso ser
    entregue.
\end{description}

Material produzido: 
relatoria (Suhellen) ,
áudio, vídeo e imagens (Reginaldo (RG) e Guine) - 

Vídeos: 
\begin{itemize}
 \item docs/relatos com os reprentantes, Pedra e Marco. (Guajará Miri)
 \item Cantiga com D.Ana Faustina ``Ama do Boi'' (representante da
   secretaria de cultura da região),bumba boi batuques e
   jogos. (Guajará Miri)
 \item relatos da trajetória da viagem (Suh, RG, Guine) -
\end{itemize}


Contatos: -Valdinei (Pedra); 9249-5913 (vivo) e 8164-2455 (tim) -Marco
Antônio T.Galiza (Cabeludo): 9224-2249 (vivo)




\chapter{Encontro em SP}


\chapter{Repositórios}\label{Repositórios}\lhead{\leftmark}

A arquitetura descentralizada do Baobáxia é baseada em repositórios
\emph{git} e \emph{git-annex}.

Um elemento importante para o acervo multimídia são as operações de
sincronização. As ferramentas baseadas no \emph{git} herdam a sua
natureza descentralizada e a capacidade de comunicar de forma
transparente usando vários protocolos. Em particular é interessante a
possibilidade de executar sincronizações com sistemas de armazenamento
massivo, característica essencial na fase de criação de um novo nó,
onde a primeira sincronização via rede poderia levar dias (ver
requisito \ref{sec:Sinc}). As transferências contudo, no caso do
\emph{git-annex}, são executadas através do protocolo
\emph{rsync}\footnote{\emph{rsync} é um Software Livre para a
  transfêrencia rápida e incremental de arquivos disponível no
  \url{http://rsync.samba.org/}.}, que gerencia eventuais
interrupções, evitando retransmissões onerosas.


\subsection{Git}\label{sec:GIT}
Git é um sistema multiplataforma para o controle de versão
distribuído, projetado para ser rápido e usável mesmo em grande
projetos.

As características principais incluem:
\begin{itemize}
\item é totalmente distribuído e cada clone de um repositório contem o
  histórico inteiro das versões e no qual podem ser efetuadas
  operações independentemente de conexões de rede o de servidores
  centrais. As mudanças podem ser copiadas entre um clone e o outro e
  são mantidos em \emph{branch} (ramos) diferentes, facilitando as
  operações de \emph{merge} (fusão). Os repositórios são facilmente
  acessíveis através do eficiente protocolo do Git, que além de
  suportar HTTP, pode funcionar junto com SSH, para obter conexões
  seguras e um sistema de autenticação solido e bem comum.
\item suporta o \emph{branching} (ramificação), e o \emph{merging}
  (fusão), de maneira rápida e conveniente, incluindo uma serie de
  ferramentas para visualizar e navegar o histórico não linear das
  versões.
\item é muito rápido e escala mesmo em projetos muito grandes e com
  muitas mudanças, graças a um eficiente sistema de empacotamento e
  memorização do histórico (é considerado o mais eficiente entre os
  sistemas atualmente disponíveis).
\item associa um nome de versão, para cada \emph{commit}, que é função
  do histórico inteiro, por isso, uma vez publicada uma versão, não é
  possível alterar as velhas sem ser notado. As versões podem também
  ser etiquetadas e assinadas digitalmente com GPG.
\end{itemize}

Git é um sistema completo que, em bom estilo Unix, é organizado em
programas e comandos independentes, pensados para ser facilmente
usáveis, seja automaticamente através de \emph{scripting} seja de
maneira interativa pelo usuário final. Git é, então, uma base solida
para o desenvolvimento de aplicações orientadas a sincronização, a
portabilidade e a gestão autônoma e descentralizada. 

\subsection{git-annex}\label{git-annex}
\emph{git-annex}\footnote{\emph{git-annex} é um programa que estende
  as funcionalidades do \emph{git} em gerir arquivos de grande tamanho
disponível no \url{http://git-annex.branchable.com}.} permite a
gestão de arquivos com \emph{git}, sem a necessidade de adicionar os
arquivos dentro \emph{git}. Mesmo se pode parecer paradoxal, é útil
quando se trabalha com arquivos muito grandes que \emph{git}
atualmente não gerencia facilmente por limitações devidas a memoria,
tempo ou espaço no disco.

Mesmo sem manter o histórico das mudanças do conteúdo do arquivo, ter
a possibilidade de gerenciar arquivos com \emph{git}, de movê-los, e
exclui-los, numa árvore de pasta versionada, com uso de
\emph{branches} e de clones distribuídos, são todos bons motivos para
usar \emph{git}. E os arquivos anexos (por isso o nome
\emph{git-annex}) podem coexistir no mesmo repositório \emph{git} com
os arquivos regularmente versionados. 

\emph{git-annex} transforma os arquivos anexos em \emph{link}
simbólicos, que são normalmente versionados por \emph{git}. 

O conteúdo dos arquivos é mantido por \emph{git-annex} em um acervo
chave/valor distribuído que corresponde aos clones de um dado
repositório \emph{git}. Praticamente \emph{git-annex} memoriza o
conteúdo do arquivo em uma subpasta de \verb|.git/annex/|.

A primeira vez que um arquivo é adicionado no \emph{git-annex}, é
calculada uma chave, normalmente fazendo um \emph{hash} do seu
conteúdo. \emph{git-annex} todavia suporta vários \emph{backend} que
podem produzir diferentes tipos de chaves. O arquivo que é adicionado
no \emph{git} nada mais é que um \emph{link} simbólico para a chave
memorizada no \verb|.git/annex/|. Se o conteúdo do arquivo for
modificado, é gerada uma outra chave, e o \emph{link} é alterado. 

O conteúdo do arquivo pode ser transferido de um repositório para
outro por \emph{git-annex}, que além de manter controle de quem mantem
o que, permite criar um mapa das copias disponíveis e impor um número
mínimo de cópias. Essas informações são mantidas em um \emph{branch}
separado, chamado ``\emph{git-annex}'', e as operações de
sincronização, são simplesmente \emph{push} e \emph{pull} entre os
vários clones dos repositórios.

\emph{git-annex} suporta:
\begin{itemize}
\item localização das cópias (\emph{location tracking})
\item download seletivo dos conteúdos 
\item gestão da confiança dos repositórios
\item gestão do número minimo de cópias
\item vários \emph{backend} para as chaves (SHA\footnote{Secure Hash
    Algoritm, (SHA), é um algoritmo usado em sistemas chave/valor onde
    as chaves são calculadas através de uma função criptográfica dos
    valores.}, WORM\footnote{O algoritmo WORM identifica os arquivos
    em base ao nome, dimensão e data de alteração.})
\item vários \emph{backend} para os conteúdos/valores
  (BUP\footnote{BUP é um sistema para \emph{backup} a alta eficiência
    disponível no: \url{https://github.com/apenwarr/bup}.}, rsync,
  web, S3\footnote{Amazon Simple Storage Service, (S3) é uma
    infraestrutura para a memorização dos dados totalmente redundante,
    disponível no: \url{aws.amazon.com/}.})
\end{itemize}


% \lstinputlisting[basicstyle={\scriptsize\ttfamily}]{../../policies/media.json}

% \subsection{Mucua}

% \lstinputlisting[basicstyle={\scriptsize\ttfamily}]{../../policies/mucua.json}


 % Multirepositorio

%\input{./Capitoli/Capitolo1} % Introduzione

%\input{./Capitoli/Capitolo2} % Reti Federate  

\clearpage  % To start a new page

%% ----------------------------------------------------------------
% The "Funny Quote Page"
\pagestyle{empty}  % No headers or footers for the following pages

\null\vfill
% Now comes the "Funny Quote", written in italics
\textit{``A força da rede esta nos nós''}

\begin{flushright}
TC
\end{flushright}

\vfill\vfill\vfill\vfill\vfill\vfill\null
\clearpage  % Funny Quote page ended, start a new page
%% ----------------------------------------------------------------

\pagestyle{fancy}  % Finally, use the "fancy" page style to implement
                   % the FancyHdr headers

%\input{./Capitoli/Capitolo4} % Prototipo

%\input{./Capitoli/Capitolo5} % Conclusioni

%\input{./Chapters/Chapter6} % Results and Discussion

%\input{./Chapters/Chapter7} % Conclusion

%% ----------------------------------------------------------------
% Now begin the Appendices, including them as separate files

\addtocontents{toc}{\vspace{2em}} % Add a gap in the Contents, for aesthetics

\appendix % Cue to tell LaTeX that the following 'chapters' are Appendices

% Apendice A                                                                                                                                                
\newcommand{\codeFolder}{../../app/django-bbx}
                   
\chapter{Listagem do código}
\label{ApendiceA}
\lhead{Apêndice A \emph{Listagem do código}}

\section{bbx}

\subsection{bbx/auth.py}
\lstinputlisting[basicstyle={\scriptsize\ttfamily}]{\codeFolder/bbx/auth.py}

\subsection{bbx/utils.py}
\lstinputlisting[basicstyle={\scriptsize\ttfamily}]{\codeFolder/bbx/utils.py}

\section{media}

%\lstinputlisting[basicstyle={\scriptsize\ttfamily}]{\codeFolder/media/README}

%\subsection{media/admin.py}
%\lstinputlisting[basicstyle={\scriptsize\ttfamily}]{../../tests/django-backbone_0/media/admin.py}

\subsection{media/models.py}
\lstinputlisting[basicstyle={\scriptsize\ttfamily}]{\codeFolder/media/models.py}

\subsection{media/serializers.py}
\lstinputlisting[basicstyle={\scriptsize\ttfamily}]{\codeFolder/media/serializers.py}

\subsection{media/views.py}
\lstinputlisting[basicstyle={\scriptsize\ttfamily}]{\codeFolder/media/views.py}

\subsection{media/urls.py}
\lstinputlisting[basicstyle={\scriptsize\ttfamily}]{\codeFolder/media/urls.py}

%\subsection{media/urls.py}
%\lstinputlisting[basicstyle={\scriptsize\ttfamily}]{../../tests/django-backbone_0/media/urls.py}

%\subsection{media/management/commands/run scheduled jobs.py}
%\lstinputlisting[basicstyle={\scriptsize\ttfamily}]{../../tests/django-backbone_0/media/management/commands/run_scheduled_jobs.py}


\section{repository}

\lstinputlisting[basicstyle={\scriptsize\ttfamily}]{\codeFolder/repository/README}

%\subsection{repository/admin.py}
%\lstinputlisting[basicstyle={\scriptsize\ttfamily}]{../../tests/django-backbone_0/repository/admin.py}

\subsection{repository/models.py}
\lstinputlisting[basicstyle={\scriptsize\ttfamily}]{\codeFolder/repository/models.py}

\subsection{repository/signals.py}
\lstinputlisting[basicstyle={\scriptsize\ttfamily}]{\codeFolder/repository/signals.py}

\subsection{repository/management/commands/run scheduled jobs.py}
\lstinputlisting[basicstyle={\scriptsize\ttfamily}]{\codeFolder/repository/management/commands/run_scheduled_jobs.py}

%\subsection{repository/admin.py}
%\lstinputlisting[basicstyle={\scriptsize\ttfamily}]{\codeFolder/repository/admin.py}

\subsection{repository/serializers.py}
\lstinputlisting[basicstyle={\scriptsize\ttfamily}]{\codeFolder/repository/serializers.py}

\section{mucua}

\subsection{mucua/models.py}
\lstinputlisting[basicstyle={\scriptsize\ttfamily}]{\codeFolder/mucua/models.py}

\subsection{mucua/serializers.py}
\lstinputlisting[basicstyle={\scriptsize\ttfamily}]{\codeFolder/mucua/serializers.py}

\subsection{mucua/views.py}
\lstinputlisting[basicstyle={\scriptsize\ttfamily}]{\codeFolder/mucua/views.py}

%\subsection{mucua/signals.py}
%lstinputlisting[basicstyle={\scriptsize\ttfamily}]{\codeFolder/mucua/signals.py}

%\subsection{mucua/urls.py}
%\lstinputlisting[basicstyle={\scriptsize\ttfamily}]{\codeFolder/mucua/urls.py}

\section{mocambola}

\subsection{mocambola/models.py}
\lstinputlisting[basicstyle={\scriptsize\ttfamily}]{\codeFolder/mocambola/models.py}

%\subsection{mocambola/serializers.py}
%\lstinputlisting[basicstyle={\scriptsize\ttfamily}]{\codeFolder/mocambola/serializers.py}

\subsection{mocambola/views.py}
\lstinputlisting[basicstyle={\scriptsize\ttfamily}]{\codeFolder/mocambola/views.py}



\section{tag}

\subsection{tag/models.py}
\lstinputlisting[basicstyle={\scriptsize\ttfamily}]{\codeFolder/tag/models.py}

\subsection{tag/serializers.py}
\lstinputlisting[basicstyle={\scriptsize\ttfamily}]{\codeFolder/tag/serializers.py}

\subsection{tag/views.py}
\lstinputlisting[basicstyle={\scriptsize\ttfamily}]{\codeFolder/tag/views.py}

%\subsection{tag/signals.py}
%\lstinputlisting[basicstyle={\scriptsize\ttfamily}]{\codeFolder/tag/signals.py}

%\subsection{tag/urls.py}
%\lstinputlisting[basicstyle={\scriptsize\ttfamily}]{\codeFolder/tag/urls.py}

\section{Interface Javascript}
\subsection{bbx/static/js/modules/bbx/router.js}
\lstinputlisting[basicstyle={\scriptsize\ttfamily}]{\codeFolder/bbx/static/js/modules/bbx/router.js}

\subsection{bbx/static/js/modules/bbx/base-functions.js}
\lstinputlisting[basicstyle={\scriptsize\ttfamily}]{\codeFolder/bbx/static/js/modules/bbx/base-functions.js}

\subsection{bbx/static/js/modules/auth/model.js}
\lstinputlisting[basicstyle={\scriptsize\ttfamily}]{\codeFolder/bbx/static/js/modules/auth/model.js}

\subsection{bbx/static/js/modules/media/model.js}
\lstinputlisting[basicstyle={\scriptsize\ttfamily}]{\codeFolder/bbx/static/js/modules/media/model.js}

\subsection{bbx/static/js/modules/media/media-functions.js}
\lstinputlisting[basicstyle={\scriptsize\ttfamily}]{\codeFolder/bbx/static/js/modules/media/media-functions.js}

\subsection{bbx/static/js/modules/mucua/router.js}
\lstinputlisting[basicstyle={\scriptsize\ttfamily}]{\codeFolder/bbx/static/js/modules/mucua/router.js}

\subsection{bbx/static/js/modules/mucua/router.js}
\lstinputlisting[basicstyle={\scriptsize\ttfamily}]{\codeFolder/bbx/static/js/modules/mucua/router.js}

\subsection{bbx/static/js/modules/mucua/model.js}
\lstinputlisting[basicstyle={\scriptsize\ttfamily}]{\codeFolder/bbx/static/js/modules/mucua/model.js}

\section{Simulação (Joey Hess)}

\subsection{simroutes.hs}
\lstinputlisting[basicstyle={\scriptsize\ttfamily}]{Apen/simroutes.hs}
	% Appendice codice sorgente

%\input{./Appendices/AppendixB} % Appendix Title

%\input{./Appendices/AppendixC} % Appendix Title

\addtocontents{toc}{\vspace{2em}}  % Add a gap in the Contents, for aesthetics
\backmatter
\nocite{*}
%% ----------------------------------------------------------------
\label{Bibliography}
\lhead{\emph{Bibliografia}}  % Change the left side page header to "Bibliography"
\bibliographystyle{unsrtnat}  % Use the "unsrtnat" BibTeX style for formatting the Bibliography
\bibliography{Bibliography}  % The references (bibliography) information are stored in the file named "Bibliography.bib"

\end{document}  % The End
%% ----------------------------------------------------------------
