\chapter{Layout e interface para usuários}\label{Layout e interface para usuários}\lhead{\leftmark}

\section{Descrição geral da pesquisa de Layout e interface}
A construção do layout e da interface compreendem do desenho à
consolidação da arquitetura geral do sistema. Layout é o desenho, a
identidade visual do sistema; por interface, entende-se o ambiente
através do qual usuários interagem com o sistema. Logo, é na
interface que se implementa a identidade visual, mas além disso, é
na interface que se define a estrutura de navegação e a disposição
das funcionalidades de acordo com o contexto de uso.

A elaboração do layout e da interface não se resumem à mera parte
visual; diz respeito a como usuários interagem com a plataforma,
como funcionalidades e recursos lhes são apresentados, além de
auxiliar na sua navegação por meio de ferramentas auxiliares.

A elaboração do desenho inicial é inicialmente apresentada e
aprovada, e em seguida aprimorada para atender às demandas do
levantamento de funcionalidades.

Em paralelo à evolução do layout, a interface deve ser estruturada
enquanto código, para dar suporte ao desenho. No caso da arquitetura
de sistema adotada, a interface se conecta à uma API, da qual
obtém dados, cabendo à primeira a apresentação dos resultados. Não
cabe à interface lidar com o armazenamento ou a lógica do sistema,
apenas com a lógica da apresentação das informações. Deste modo,
a construção da \emph{API} e a estruturação da plataforma precedem o
desenvolvimento da interface.

Entretanto, por se tratar de uma ferramenta experimental e em fase
de desenvolvimento e maturação, conforme a construção da interface
avançou, foi possível detectar alguns problemas conceituais, além
da incorporação de novas funcionalidades. Logo, na criação do
software Baobáxia, fugiu-se à lógica tradicional de análise de
sistemas compartimentada em fases rígidas. A elaboração do produto
do layout e da interface rendeu sugestões de aprimoramentos na API,
alguns dos quais já implementados, outros anotados para posterior
desenvolvimento.

\section{Levantamento de funcionalidades}
O levantamento das funcionalidades do sistema foi a primeira etapa
realizada para criação do layout e implantação da interface. Para
tal, contamos com levantamentos textuais e na elaboração de um
\emph{wireframe}.

\subsection{Wireframe}
O desenho de wireframe é um método eficiente para o design de
interfaces, ilustrando as páginas dos sistema, suas funcionalidades
e o fluxo de interação entre cada uma delas. Por meio do wireframe
os trabalhos para construção da identidade visual são orientados,
de modo que o desenho seja compatível com a estrutura do sistema.

O wireframe completo está disponível em: 
\url{http://git.io/K1dJ0Q}.
O arquivo pode ser visualizado utilizando o programa Inkscape \footnote{disponível
  em: \url{http://inkscape.org}.}.

Foram listadas no \emph{wireframe} as seguintes páginas:
\begin{itemize}
  \item login de usuário
  \item home da mucua
  \item visualização da rede
  \item visualização por tag
  \item visualização por tags combinadas
  \item visualização no modo grid
  \item visualização no modo lista
  \item página do arquivo
  \item página do arquivo (visualzada a partir de outra mucua com alta
    conectividade
  \item página do arquivo (visualzada a partir de outra mucua com baixa
    conectividade
  \item envio de arquivos
  \item edição de arquivos
  \item listagem de mucuas
  \item informações da mucua
  \item sincronização de dispositivo
\end{itemize}

Além destas, ainda constam como funcionalidades/ações:
\begin{itemize}
  \item registrar
  \item logout
  \item esqueci a senha
  \item página do mocambola / profile
  \item edição dos dados do mocambola / profile
  \item edição da mucua
  \item lista de requisições da mucua
  \item informações da rede
  \item cabeçalho
  \item rodapé com indicação de uso da mucua
\end{itemize}    

\subsection{Arquitetura}
A arquitetura da interface é modular, conectando-se com a \emph{API}
\footnote{Uma \emph{API}\footnote{Application programming interface}
  especifica um componente de software em termos de suas operações,
  entradas e saídas e coorrespondentes tipos. Seu objetivo é definir
  um conjunto de funcionalidades que são independentes da implementação
  adotada, permitindo que tanto a implementação técnica (para dentro)
  como a interface (que se encaixa) sejam alteradas, sem dano para
  qualquer uma das partes.} do software básico. Isso quer dizer que
é possível desenvolver outras interfaces em cima da mesma \emph{API}
- como uma interface em modo texto, ou uma interface exclusiva para
telefones celulares - sem precisar reprogramar o núcleo do programa.

A arquitetura baseada em \emph{APIs} é perfeitamente adequada para
softwares que se projetam como de longa duração, já que alterações
tanto no interior do sistema como na interface podem ser feitas sem
prejuízo para as outras partes. Uma melhoria interna no núcleo do sistema
pode ser implementada sem que nem sequer se perceba alteração na parte
da interface; do mesmo modo, qualquer alteração e implementação nova da
interface pode ser feita contanto que utilize as rotas da \emph{API}
existente. Trata-se de uma arquitetura modular, desacoplada,
perfeitamente adequada para a natureza de um sistema como o Baobáxia.

A \emph{API} já encontrava-se desenhada e em parte implementada; mas
durante a construção da interface - a parte visual do sistema - novas
tarefas e funcionalidades, expressa em rotas e \emph{urls}, foram
adicionadas, chegando-se a um desenho mais próximo de uma versão
operacional do sistema. 

O desenho da arquitetura da \emph{API} é fundamental para construção
da interface, pois é por meio das urls da \emph{API} que o sistema
irá buscar informações de conteúdo e enviar dados para realizar ações.

As rotas listadas abaixo estão divididas por áreas

\emph{Autenticação}
\begin{itemize}
  \item /login (funcionalidade de login)
  \item /[repo]/[mucua]/login (funcionalidade de login - com caminho)
  \item /logout (funcionalidade de logout)
  \item /[repo]/[mucua]/logout (funcionalidade de login - com caminho)
  \item /register (registro de novos usuários)
  \item /[repo]/[mucua]/register (registro de novos usuários - com caminho)
  \item /lost\_password (recuperação de senha)
  \item /[repo]/[mucua]/lost\_password (recuperação de senha - com caminho) 
\end{itemize}

\emph{Mocambola}
\begin{itemize}
  \item /[repo]/[mucua]/mocambola/[user] {get}          profile do mocambola
  \item /[repo]/[mucua]/mocambola/[user] {put, delete}  edição de dados do mocambola
\end{itemize}

\emph{Mucua}
\begin{itemize}
  \item /[repo]/[mucua] {get}         home da mucua
  \item /[repo]/[mucua] {put}         atualização da mucua
  \item /[repo]/[mucua]/info          informações sobre a mucua
  \item /[repo]/[mucua]/requests      requisições da mucua
\end{itemize}

\emph{Home da rede}
\begin{itemize}                
  \item /rede                         home da rede
  \item /rede/info                    informações da rede
\end{itemize}

\emph{Funcionalidades media}
\begin{itemize}
  \item /[repo]/[mucua]/media {post}                inserir media
  \item /[repo]/[mucua]/media/[uuid] {get}          retornar media
  \item /[repo]/[mucua]/media/[uuid] {put, delete}  atualizar media
  \item /[repo]/[mucua]/media/last {get}            retorna últimas medias adicionadas
  \item /[repo]/[mucua]/media/[uuid]/url            retorna url do arquivo
  \item /[repo]/[mucua]/media/[uuid]/[width]x[height].[format]       retorna imagem da media
  \item /[repo]/[mucua]/media/[uuid]/related        retorna media relacionada à media atual [uuid]
\end{itemize}

\emph{Funcionalidades básicas para definição de escopo}
\begin{itemize}
  \item /repository/list                     lista repositórios disponíveis
  \item /repository/*                        retorna repositório padrão
  \item /[repo]/mucuas                       retorna mucuas do [repositório]
  \item /mucua/                              retorna mucua padrão
  \item /mucua/list                          retorna lista das mucuas
\end{itemize}

\emph{Busca}
\begin{itemize}
  \item /[repo]/[mucua]/bbx/search/[arg1]/[arg2]/...  busca por argumentos
\end{itemize}

\section{Layout da interface}
Por Layout entendemos o desenho propriamente dito, na forma de uma
imagem, a ser implementada pelo programador web. O trabalho divide-se
entre a criação de uma identidade visual (ideia) e o desenho das
telas principais (a aplicação da ideia a casos reais).

\subsection{Identidade visual}

\emph{Baobá, árvore que vive milhares de anos e representa simbolicamente
a memória coletiva ligada ao território. Baobáxia é a união de baobá com
galáxia. Uma galáxia de memórias coletivas locais ligadas ao território,
criando caminhos entre as estrelas.}

O desenvolvimento da interface gráfica e da identidade visual do Baobaxia
visa servir de meio visual e dar forma e expressão a construção de uma
nova rede, que interconecte espaços quilombolas e proporcione a troca de
conteúdo de maneira simples e neutra, sem requisições de passagens por
outras redes e prezando a troca de saberes entre as comunidades.

A identidade visual do Baobáxia buscou trazar a identidade da cultura livre,
dos povos quilombolas, da cultura africana e da simbologia que se criou com
a construção do termo Baobáxia e suas origens, funções e mistérios: o baobá
e as galáxias. Trazer o sentimento de fazer parte de um todo que cria,
transforma, compartilha e se fortalece no território virtual e real.

A interface gráfica busca provocar o sentimento de autonomia, estabilidade
e troca nos usuários do aplicativo Baobáxia. Buscando expandir, transformar
e espalhar essa tecnologia social, com o importante papel de verdadeiramente
comunicar e estimular uma interação ativa, criadora e consciente do
aplicativo. 

%% \begin{figure}[htbp]
%%   \centering
%%   \includegraphics[width=\textwidth]{./Fig/Logo_BBX.pdf}
%%   \rule{35em}{0.5pt}
%%   \caption[Logomarca Baobáxia]{Logomarca Baobáxia}
%%   \label{fig:Logo_BBX}
%% \end{figure}

\begin{figure}[htbp]
  \centering
  \includegraphics[width=\textwidth]{./Fig/Logo_BBX_Na_Rota.pdf}
  \rule{35em}{0.5pt}
  \caption[Logomarca Baobáxia na Rota dos Baobás]{
    Logomarca Baobáxia na Rota dos Baobás}
  \label{fig:Logo_BBX_Na_Rota}
\end{figure}

\begin{figure}[htbp]
  \centering
  \includegraphics[width=\textwidth]{./Fig/layout-login.pdf}
  \rule{35em}{0.5pt}
  \caption[Página de login]{Página de login}
  \label{fig:layout-login}
\end{figure}

\begin{figure}[htbp]
  \centering
  \includegraphics[width=\textwidth]{./Fig/layout-pgCONTEUDO.pdf}
  \rule{35em}{0.5pt}
  \caption[Página do conteúdo]{Página do conteúdo}
  \label{fig:layout-pgCONTEUDO}
\end{figure}

\begin{figure}[htbp]
  \centering
  \includegraphics[width=\textwidth]{./Fig/layout-pgMUCUA.pdf}
  \rule{35em}{0.5pt}
  \caption[Página da Mucua]{Página da Mucua}
  \label{fig:layout-pgMUCUA}
\end{figure}

\begin{figure}[htbp]
  \centering
  \includegraphics[width=\textwidth]{./Fig/layout-pgREDE.pdf}
  \rule{35em}{0.5pt}
  \caption[Página da Rede]{Página da Rede}
  \label{fig:layout-pgREDE}
\end{figure}

\begin{figure}[htbp]
  \centering
  \includegraphics[width=\textwidth]{./Fig/layout-pgREDEbusca.pdf}
  \rule{35em}{0.5pt}
  \caption[Página da Rede com busca]{Página da Rede com busca}
  \label{fig:layout-pgREDEbusca}
\end{figure}

\subsection{Desenho das telas principais}

Após elaborada a identidade visual, partiu-se para o desenho das
principais telas do sistema. A partir delas, o trabalho de codificação
da interface pode ser iniciado, e também ajustes no desenho puderam ser
feitos após a implementação em código do desenho.

Decidiu-se que as principais telas para a fase inicial do projeto seriam:
\begin{itemize}
  \item Página inicial / tela de login
  \item Página inicial da mucua
  \item Página inicial da rede
  \item Rodapé com indicativos sobre utilização da mucua
  \item Interface de busca combinada
  \item Página do conteúdo
\end{itemize}

\newpage
\newpage
\newpage

%%
%% INTERFACE PARA USUÁRIOS
%%


\section{Interface para usuários}
A interface propriamente dita é a interação do sistema com o usuário;
deve conter as funcionalidades básicas do sistema disponíveis para
interação, implementada na forma de uma página de internet. O presente
tópico pretende discorrer sobre as tecnologias e a arquitetura adotadas.


\subsection{Arquitetura da interface}
A interface do Baobáxia é a parte superior, mais aparente, do sistema,
conversando com o seu núcleo por meio de uma \emph{API}. Decidiu-se
pela implementação de uma interface em HTML e JavaScript, utilizando
uma série de bibliotecas de código, buscando separar as camadas do modelo
de dados, lógica de programação, roteamento, visualização dos dados e
templates de páginas.

Para garantir tal arquitetura, adotamos as seguintes bibliotecas:
\begin{itemize}
  \item \emph{Backbone.Js}: para lidar com o modelo de dados e as conexões
    com a \emph{API}, cuidando da atualização dos dados
  \item \emph{Require.js}: cuida do carregamento dinâmico das bibliotecas
    e suas dependências, adicionando tal funcionalidade ao JavaScript
  \item \emph{Lodash.js}: biblioteca assistente para o código, adiciona
    uma série de métodos úteis ao trato com os dados e a exibição dos
    conteúdos na forma de Templates
  \item \emph{Jquery} e outras bibliotecas auxiliares de código, para
    manipulação dos dados do \emph{DOM} \footnote Document Object Model
\end{itemize}

\subsection{Arquitetura de código da interface}
A atual implementação da interface localiza-se na pasta de conteúdo estático
do Django, localizada em:
\emph{/srv/bbx/baobaxia/app/django-bbx/bbx/static}

Note que esta é a pasta raiz para operações na interface. Listando o conteúdo,
temos:
\begin{itemize}
  \item /css     -> arquivos de formatação de estilos html
  \item /images  -> imagens estáticas usadas pela interface
  \item index.html  -> página inicial
  \item /js      -> aplicação JavaScript
\end{itemize}

Sob a pasta /js, temos outra estrutura, que obedece a certos padrões de código
por nós definidos:
\begin{itemize}
  \item app.js   -> definição do aplicativo
  \item config.json  -> arquivo de configuração da interface
  \item config.json.example  -> exemplo de arquivo de configuração da interface
  \item /lib     -> localização de todas as bibliotecas javascript utilizadas pela interface
  \item main.js  -> função inicializadora do aplicativo
  \item /modules  -> módulos Backbone.js
  \item README    -> Arquivo de instruções
  \item router.js   -> roteamento da aplicação
  \item /views    -> visualizações de dados da interface
\end{itemize}

Dentre as bibliotecas utilizadas, temos as seguintes:
\begin{itemize}
  \item backbone-amd.js   -> biblioteca Backbone.js
  \item backbone.subroute.min.js -> sub rotas para o Backbone.js
  \item jquery.cookie.js  -> manipulação de cookies via jQuery
  \item jquery.json.min.js  -> manipulação de arquivos JSON via jQuery
  \item jquery-min.js     -> bibloteca jQuery
  \item jquery.tagcloud.js   -> geração de nuvem de tags via jQuery
  \item lodash-min.js    -> biblioteca acessória para JavaScript, substituição para Underscore
  \item require    -> plugins da biblioteca require (json.js e text.js)
  \item require.js  -> biblioteca Require.js
  \item textext -> biblioteca jQuery para campos de formulário
\end{itemize}

\textbf{Backbone.js}\\
O Backbone.js é uma biblioteca JavaScript para criar uma estrutura de
aplicação compatível com chamadas a APIs REST que retornam resultados em JSON.
Foi adotada por apresentar boa comunicação com a API de Backend do Baobáxia, baseado
no Django REST Framework. Trabalha com modelos (models), coleções (collections) e
visualizações (views). Por meio dele, foi possível construir uma interface
baseada em eventos e assíncrona; ou seja, a interface é renderizada conforme são
carregados os elementos a partir da API.

Mais informações sobre o BackboneJS podem ser obtidas na sua página: \url{http://backbonejs.org}

\textbf{jQuery}\\
JQuery é uma biblioteca consolidada para manipulação de elementos HTML via
JavaScript. Por meio dela, é possível alterar estilos, textos e remodelar completamente
uma página HTML. É possível manipular campos identificados por um ID, por classes
de css ou mesmo diretamente os elementos HTML, como no exemplo abaixo:

\begin{code}
  <p>Título</p>
  <div id="id-campo">texto velho</div>
  <div id="id-campo2">algum texto... </div>
  <div class="css-class"></div>
  <p>Outro texto...</p>

  <script type="text/javascript">
  $('#id-campo').html('novo texto');
  $('#id-campo2').append('adicionando o link: <a href="http://www.url.com">link</a>');
  $('.css-class').html('<a href="http://www.url.com">link</a>');
  $('p').css('font-weight: bold');
  </script>
\end{code}

O resultado será algo como:
\\
\emph{Título}\\
novo texto\\
algum texto... adicionando o link: \href{http://www.url.com}{link}\\
\emph{Outro texto...}

Mais informações sobre o jQuery podem ser obtidas em: \url{http://jquery.com/}

\textbf{Require.js}\\
A biblioteca Require.js cumpre o propósito de carregar arquivos e módulos de uma
aplicação JavaScript, garantindo que a execução do código só se dê após finalizado
o carregamento. Permite que bibliotecas sejam definidas como base para outras,
garantindo o carregamento primeiro das dependências para então carregar o arquivo
em questão.

Usa-se um arquivo de definição geral dos caminhos, localizado em main.js. Por exemplo:
\begin{code}
require.config({
    shin: {
	lodash: { 
	    exports: '_'
	},
	backbone: {
	    deps: ['lodash', 'jquery'],
	    exports: 'Backbone'
	},
    },
    paths: {
	jquery: 'lib/jquery-min',
	lodash: 'lib/lodash-min',
	backbone: 'lib/backbone-amd',
    },
    waitSeconds: 200
});

\end{code}

Em outro arquivo, podem ser carregados:

\begin{code}
  define([
    'backbone',
    'jquery', 
    'lodash',
  ], function(Backbone, $, _) {
    ...
  }
\end{code}

Neste caso, o jQuery possui apenas um carregamento simples. Com a chave
'shim' definida, são passados atributos especiais; no caso do Lodash,
apenas o arquivo é exportado para a variável \_. No caso do Backbone,
além de definir um objeto a ser exportado, são apresentadas suas dependências,
no caso, do Lodash e do JQuery. Deste modo, o arquivo do Backbone só será
carregado após o correto carregamento das dependências, independentemente
da ordem em que forem posicionados no arquivo que for carregá-los.

Mais informações podem ser encontradas em: \url{http://requirejs.org/}

Além das funcionalidades oferecidas para o programador JavaScript, pelo Require.js
é possível incluir arquivos do tipo JSON ou texto, o que pode ser utilizado para
renderizar modelos html (templates) por meio do Underscore ou Lodash. Para tal,
utilizamos o plugin de texto, localizado em 'lib/require/text.js'.

Por meio do Lodash, implementamos um sistema de templates para a interface. O
Lodash possui a função \_.template(text, data, options), que pode ser utilizada
com o template abaixo:

\begin{code}
  <div id="media-view">
  <% if (username) { %>
  <div id="link-edit">
    <a href="<%= '#' + media.repository + '/' + media.origin +  '/media/' + media.uuid %>/edit">
      <img src="images/edit-media.png" />
    </a>
  </div>
  <% } %>

  <div id="media-title">
      <h3><%= nome %></h3>
  </div>
\end{code}

Para renderizar o template, chame a função e passe os dados:

\begin{code}

define([
    'jquery', 
    'lodash',
    'backbone', 
      'text!templates/media/MediaView.html'
], function($, _, Backbone, MediaViewTpl){
      var MediaView = Backbone.View.extend({
	render: function(){
          var data = {
            username: 'fulano@tal.org',
            nome: 'fulano'
          }
          $('#content').html(_.template(MediaViewTpl, data);
        }
      });
      ...            
\end{code}

É possível adicionar lógica a partir das variáveis que são passadas ao template,
criando um "html dinâmico".

\textbf{Lodash}\\
Lodash é uma biblioteca derivada do Underscore.js (\url{http://underscorejs.org/}),
que expande suas funcionalidades e tem suporte nativo à especificação AMD (Asynchronous
module definition - definição de módulos assíncrona), para carregamento de móduls
em JavaScript.

Adicionalmente, a biblioteca Lodash expande as funcionalidades do Underscore.js,
com mais funções para manipulação de matrizes, objetos e estruturas de dados.

Como o Underscore, o Lodash também utiliza o caractere como padrão \_ para suas
funcionalidades. Desse modo, uma vez carregado, podemos:

\begin{code}
  <script text='text/javascript'>
  _.first(['abc', 'def', 'ghi']);    // retornará 'abc'
  </script>
\end{code}

Mais informações podem ser encontradas em: \url{https://lodash.com/}


A combinação das bibliotecas acima criou um ambiente de desenvolvimento flexível,
que permite que novas rotas ao usuário sejam adicionadas, apontando para funções
internas, chamadas de API etc.

Um módulo, sob a pasta /modules/[nome], pode ter os seguintes arquivos:
\begin {itemize}
  \item model.js    -> modelo de dados; arquivo necessário ao Backbone.js
  \item collection.js   -> coleção de modelos; arquivo necessário ao Backbone.js para armazenar conjuntos de dados
  \item functions.js   -> funções específicas do módulo
  \item router.js   -> arquivo de sub rotas
\end{itemize}

No arquivo \emph{/srv/bbx/baobaxia/app/django-bbx/bbx/static/js/router.js} são
adicionadas rotas simples e rotas complexas. Abaixo, segue o padrão 'caminho' : 'nomeDaFuncao',
como no exemplo abaixo:

\begin{code}
  routes: {
    'logout': 'logout',
    ':repository/:mucua/logout': 'logout',
    ':repository/:mucua/bbx/*subroute': 'invokeBbxModule',
  },
  
  logout: function(repository, mucua) {
    ...
  },
  
  invokeMucuaModule: function(repository, mucua, subroute) {
    var subroute = subroute || '';
    ...
    this.Routers.MucuaRouter = new MucuaRouter(repository + "/" + mucua + "/", subroute);
  }      
\end{code}

Nesse caso, logout será encaminhada para a função 'logout', e todas as sub-rotas dentro
de /[repository]/[mucua]/mucua/bbx/ serão encaminhadas para outro arquivo de rotas,
de sub rotas associadas ao módulo (no caso, 'mucua'). Esse arquivo de rotas deve ser
posicionado em \emph{/srv/bbx/baobaxia/app/django-bbx/bbx/static/js/modules/nome\_do\_modulo/router.js}

O código terá a instrução de rotas com a seguinte sintaxe:

\begin{code}
  routes: {
    '*' : 'homeMucua',
    'info' : 'infoMucua',
  },
  homeMucua: function() {
    ...
  },
  infoMucua: function() {
    ...
  }
\end{code}

Adicionamos à estrutura de arquivos básica do Backbone.js arquivos de funções gerais,
localizados na pasta /modules; deste modo, caso um módulo tenha funções
específicas, estas deve estar contidas num arquivo \emph{functions.js} na
respectiva pasta /modules. Ex:

\begin{itemize}
  \item /srv/bbx/baobaxia/app/django-bbx/bbx/static/js/modules/bbx/functions.js
  \item /srv/bbx/baobaxia/app/django-bbx/bbx/static/js/modules/media/functions.js
\end{itemize}

As visualizações (views) são chamadas pelas rotas (router.js), articulando modelos
(models.js) e dando saída por meio de templates.

\subsection{Implementação da interface junto à API}
Foi montada uma estrutura que permite a fácil e rápida expansão das
funcionalidades do sistema na interface, obedecendo a um padrão de rotas e
disposição de funcionalidades. A interface se montou acompanhando a \emph{API},
surgerindo novas funcionalidades ainda não implementadas no núcleo do sistema.

A estrutura é a que segue:
\begin{itemize}
  \item definição de bibliotecas e mapeamento de endereços
  \item funcionalidades relacionadas à media
  \item funcionalidades relacionadas à tarefas do Baobáxia
  \item visualizações (autenticação, media, mocambola, mucua e comuns)
  \item templates (modelos de dados para autenticação, media, mocambola e mucua)
  \item estilos (formatação HTML / arquivos css)
\end{itemize}

\subsection{Implementação das telas principais}
Foram implementadas funcionalidades nas seguintes frentes:

\textbf{Autenticação e usuários}\\

Implementou-se um sistema básico de login, com logout e cadastro de novos
usuários, todo pela interface. O usuário pode também publicar conteúdos e
editar os existentes.

\textbf{Mucuas}\\

Foi criada uma página para a Mucua, com bloco de destaques e conteúdo recente.
É possível fazer buscas na mucua e obter informações gerais sobre a mucua,
sobre uso do disco etc.

\textbf{Rede}\\

Sempre é exibida uma aba para a Rede, que alterna a busca atual para o escopo
da rede. Assim, uma busca geral na mucua alternará para uma busca geral na rede;
uma busca por 'video' na mucua alternará para uma busca de 'video' na rede toda.

Por rede, entende-se o conteúdo que é difundido por todas as mucuas, incluída a
mucua atual.

Sob a consulta de Rede, é possível ver a lista das mucuas conectadas ao
repositório, com link para cada uma delas.

\textbf{Buscas}\\

Os conteúdos são buscados em vários campos:

\begin{itemize}
  \item título
  \item descrição
  \item tipo de conteúdo
  \item formato
  \item categorias (em implantação)    
\end{itemize}

É possível listar conteúdos por Mucua e por Mocambola.

\textbf{Listagens}\\

São oferecidas listagens no formato grid, com possibilidade de executar ou
visualizar o conteúdo na própria página; ou no formato listagem, com possibilidade
de ordenar os conteúdos por título, autor, data de publicação, origem, licença, tipo,
número de cópias pela rede e se há cópia do arquivo na mucua local.

\begin{figure}[htbp]
  \centering
  \includegraphics[width=\textwidth]{./Fig/InterfaceBaobaxia-busca-video.pdf}
  \rule{35em}{0.5pt}
  \caption[Busca por vídeo + baobá]{Busca por vídeo + baobá}
  \label{fig:InterfaceBaobaxia-busca-video}
\end{figure}

\begin{figure}[htbp]
  \centering
  \includegraphics[width=\textwidth]{./Fig/InterfaceBaobaxia-busca-video-baoba.pdf}
  \rule{35em}{0.5pt}
  \caption[Busca por vídeo + baobá]{Busca por vídeo + baobá}
  \label{fig:InterfaceBaobaxia-busca-video-baoba}
\end{figure}

\begin{figure}[htbp]
  \centering
  \includegraphics[width=\textwidth]{./Fig/InterfaceBaobaxia-home-mucua.pdf}
  \rule{35em}{0.5pt}
  \caption[Home da mucua]{Home da mucua}
  \label{fig:InterfaceBaobaxia-home-mucua}
\end{figure}

\begin{figure}[htbp]
  \centering
  \includegraphics[width=\textwidth]{./Fig/InterfaceBaobaxia-listagem-sort.pdf}
  \rule{35em}{0.5pt}
  \caption[Busca no modo listagem, com ordernação]{Busca no modo listagem, com ordernação}
  \label{fig:InterfaceBaobaxia-listagem-sort}
\end{figure}

\begin{figure}[htbp]
  \centering
  \includegraphics[width=\textwidth]{./Fig/InterfaceBaobaxia-lista-mucuas.pdf}
  \rule{35em}{0.5pt}
  \caption[Lista de mucuas]{Lista de mucuas}
  \label{fig:InterfaceBaobaxia-lista-mucuas}
\end{figure}

\begin{figure}[htbp]
  \centering
  \includegraphics[width=\textwidth]{./Fig/InterfaceBaobaxia-login.pdf}
  \rule{35em}{0.5pt}
  \caption[Tela de login]{Tela de login}
  \label{fig:InterfaceBaobaxia-login}
\end{figure}

\begin{figure}[htbp]
  \centering
  \includegraphics[width=\textwidth]{./Fig/InterfaceBaobaxia-media-view.pdf}
  \rule{35em}{0.5pt}
  \caption[Visualização de mídia]{Visualização de mídia}
  \label{fig:InterfaceBaobaxia-media-view}
\end{figure}

\begin{figure}[htbp]
  \centering
  \includegraphics[width=\textwidth]{./Fig/InterfaceBaobaxia-media-publish.pdf}
  \rule{35em}{0.5pt}
  \caption[Publicação de mídia]{Publicação de mídia}
  \label{fig:InterfaceBaobaxia-media-publish}
\end{figure}

\begin{figure}[htbp]
  \centering
  \includegraphics[width=\textwidth]{./Fig/InterfaceBaobaxia-media-edit.pdf}
  \rule{35em}{0.5pt}
  \caption[Edição de mídia]{Edição de mídia}
  \label{fig:InterfaceBaobaxia-media-edit}
\end{figure}

\begin{figure}[htbp]
  \centering
  \includegraphics[width=\textwidth]{./Fig/InterfaceBaobaxia-mucua-listagem.pdf}
  \rule{35em}{0.5pt}
  \caption[Lista de mucuas]{Lista de mucuas}
  \label{fig:InterfaceBaobaxia-mucua-listagem}
\end{figure}

\begin{figure}[htbp]
  \centering
  \includegraphics[width=\textwidth]{./Fig/InterfaceBaobaxia-rede-listagem.pdf}
  \rule{35em}{0.5pt}
  \caption[Listagem em Rede]{Listagem em Rede}
  \label{fig:InterfaceBaobaxia-rede-listagem}
\end{figure}

\begin{figure}[htbp]
  \centering
  \includegraphics[width=\textwidth]{./Fig/InterfaceBaobaxia-registrar.pdf}
  \rule{35em}{0.5pt}
  \caption[Registro de novos usuários]{Registro de novos usuários}
  \label{fig:InterfaceBaobaxia-registrar}
\end{figure}

