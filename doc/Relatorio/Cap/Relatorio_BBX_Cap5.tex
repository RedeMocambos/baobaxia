\chapter{Layout e interface para usuários}\label{Layout e interface para usuários}\lhead{\leftmark}

\section{Descrição geral da pesquisa de Layout e interface}
A construção do layout e da interface compreendem do desenho à
consolidação da arquitetura geral do sistema. Layout é o desenho, a
identidade visual do sistema; por interface, entende-se o ambiente
através do qual usuários interagem com o sistema. Logo, é na
interface que se implementa a identidade visual, mas além disso, é
na interface que se define a estrutura de navegação e a disposição
das funcionalidades de acordo com o contexto de uso.

A elaboração do layout e da interface não se resumem à mera parte
visual; diz respeito a como usuários interagem com a plataforma,
como funcionalidades e recursos lhes são apresentados, além de
auxiliar na sua navegação por meio de ferramentas auxiliares.

A elaboração do desenho inicial é inicialmente apresentada e
aprovada, e em seguida aprimorada para atender às demandas do
levantamento de funcionalidades.

Em paralelo à evolução do layout, a interface deve ser estruturada
enquanto código, para dar suporte ao desenho. No caso da arquitetura
de sistema adotada, a interface se conecta à uma API, da qual
obtém dados, cabendo à primeira a apresentação dos resultados. Não
cabe à interface lidar com o armazenamento ou a lógica do sistema,
apenas com a lógica da apresentação das informações. Deste modo,
a construção da API e a estruturação da plataforma precedem o
desenvolvimento da interface.

Entretanto, por se tratar de uma ferramenta experimental e em fase
de desenvolvimento e maturação, conforme a construção da interface
avançou, foi possível detectar alguns problemas conceituais, além
da incorporação de novas funcionalidades. Logo, na criação do
software Baobáxia, fugiu-se à lógica tradicional de análise de
sistemas compartimentada em fases rígidas. A elaboração do produto
do layout e da interface rendeu sugestões de aprimoramentos na API,
alguns dos quais já implementados, outros anotados para posterior
desenvolvimento.

\section{Levantamento de funcionalidades}
O levantamento das funcionalidades do sistema foi a primeira etapa
realizada para criação do layout e implantação da interface. Para
tal, contamos com levantamentos textuais e na elaboração de um
wireframe.

\subsection{Wireframe}
O desenho de wireframe é um método eficiente para o design de
interfaces, ilustrando as páginas dos sistema, suas funcionalidades
e o fluxo de interação entre cada uma delas. Por meio do wireframe
os trabalhos para construção da identidade visual são orientados,
de modo que o desenho seja compatível com a estrutura do sistema.

O wireframe completo está disponível em: https://raw.githubusercontent.com/RedeMocambos/baobaxia/master/doc/layout/wireframe.svg.
O arquivo pode ser executado utilizando o programa Inkscape (disponível
em: http://inkscape.org).

Foram listadas as seguintes páginas:
\begin{itemize}
\item login de usuário
\item home da mucua
\item visualização da rede
\item visualização por tag
\item visualização por tags combinadas
\item visualização no modo grid
\item visualização no modo lista
\item página do arquivo
\item página do arquivo (visualzada a partir de outra mucua com alta
  conectividade
\item página do arquivo (visualzada a partir de outra mucua com baixa
  conectividade
\item envio de arquivos
\item edição de arquivos
\item listagem de mucuas
\item informações da mucua
\item sincronização de dispositivo
\end{itemize}

\subsection{Arquitetura}

\section{Layout da interface}
\subsection{Identidade visual}
\subsection{Desenho das telas principais}

\section{Interface para usuários}
\subsection{Arquitetura da interface}
\subsection{Implementação da interface junto à API}
\subsection{Implementação das telas principais}
\subsection{Diagnóstico de problemas, alterações e ajustes no layout}
