\chapter{Transferências e sincronização dos mídias}\label{Transferências e sincronização dos mídias}\lhead{\leftmark}

\section{Transferência agendada e seletiva de arquivos}
A arquitetura do \emph{git-annex} se baseia em clones do repositório
que sincronizam entre si os conteúdos. Pela especifica de requisitos
da RM é necessário que as operações de sincronização possam acontecer
em contextos offline. Nesses contextos os conteúdos de fato viagiam
fisicamente em suportes quais laptops, HDs e pendrive. O meio de
transporte dos dados é portanto baseado nas rotas fisicas da Rota dos
Baobás. As Rotas nascem de vinculos concretos ancestrais, de
afinidade, troca de produtos, encontros e vivencias, que já existem
entre as comunidades. O Baobáxia tenta aproveitar o maximo possível as
logicas e os vinculos preexistentes. Para entender melhor o
funcionamento detalhamos um pouco a Rota dos Baobás a partir da Tainã
que mantem relações constantes com:
\begin{itemize}
  \item Quilombo do Cafundó
  \item Fazenda Roseira
  \item Quilombo de Brotas
  \item Mercado Sul
\end{itemize}

A Tainã e o Mercado Sul dispõem de conexão em banda larga, a Roseira
esta sem conexão internet e o Cafundó tem uma conexão satelitar.

A base da logica de triagem e circulação dos conteúdos é gerenciada
atraves dos ``preferred contents'' do \emph{git-annex}, que
possibilita organizar as \emph{mucuas} por grupos e definir regras de
circulação entre esses grupos baseados em diferentes metadados (tipo
de arquivo, tag, tamanho, ...).

A rede do Baobáxia pode ser reconfigurada ``em andamento'', ou seja as
logica de triagem podem mudar no tempo. Escolhemos como primeira
configuração:

\begin{tabular}{ l | c || r }
  \hline                       
  nucleo & 3 & taina, mercado sul \\
  sync & 2 & raspberry, laptop \\
  online & 2 & acotirene,  \\
  \hline  
\end{tabular}




\section{Gestão de conflitos entre versões}

%% A arquitetura descentralizada do Baobáxia é baseada em repositórios
%% \emph{git} e \emph{git-annex}.

%% Um elemento importante para o acervo multimídia são as operações de
%% sincronização. As ferramentas baseadas no \emph{git} herdam a sua
%% natureza descentralizada e a capacidade de comunicar de forma
%% transparente usando vários protocolos. Em particular é interessante a
%% possibilidade de executar sincronizações com sistemas de armazenamento
%% massivo, característica essencial na fase de criação de um novo nó,
%% onde a primeira sincronização via rede poderia levar dias. As
%% transferências contudo, no caso do \emph{git-annex}, são executadas
%% através do protocolo \emph{rsync}\footnote{\emph{rsync} é um Software
%%   Livre para a transfêrencia rápida e incremental de arquivos
%%   disponível no \url{http://rsync.samba.org/}.}, que gerencia
%% eventuais interrupções, evitando retransmissões onerosas.


%% \subsection{Git}\label{sec:GIT}
%% Git é um sistema multiplataforma para o controle de versão
%% distribuído, projetado para ser rápido e usável mesmo em grande
%% projetos.

%% As características principais incluem:
%% \begin{itemize}
%% \item é totalmente distribuído e cada clone de um repositório contem o
%%   histórico inteiro das versões e no qual podem ser efetuadas
%%   operações independentemente de conexões de rede o de servidores
%%   centrais. As mudanças podem ser copiadas entre um clone e o outro e
%%   são mantidos em \emph{branch} (ramos) diferentes, facilitando as
%%   operações de \emph{merge} (fusão). Os repositórios são facilmente
%%   acessíveis através do eficiente protocolo do Git, que além de
%%   suportar HTTP, pode funcionar junto com SSH, para obter conexões
%%   seguras e um sistema de autenticação solido e bem comum.
%% \item suporta o \emph{branching} (ramificação), e o \emph{merging}
%%   (fusão), de maneira rápida e conveniente, incluindo uma serie de
%%   ferramentas para visualizar e navegar o histórico não linear das
%%   versões.
%% \item é muito rápido e escala mesmo em projetos muito grandes e com
%%   muitas mudanças, graças a um eficiente sistema de empacotamento e
%%   memorização do histórico (é considerado o mais eficiente entre os
%%   sistemas atualmente disponíveis).
%% \item associa um nome de versão, para cada \emph{commit}, que é função
%%   do histórico inteiro, por isso, uma vez publicada uma versão, não é
%%   possível alterar as velhas sem ser notado. As versões podem também
%%   ser etiquetadas e assinadas digitalmente com GPG.
%% \end{itemize}

%% Git é um sistema completo que, em bom estilo Unix, é organizado em
%% programas e comandos independentes, pensados para ser facilmente
%% usáveis, seja automaticamente através de \emph{scripting} seja de
%% maneira interativa pelo usuário final. Git é, então, uma base solida
%% para o desenvolvimento de aplicações orientadas a sincronização, a
%% portabilidade e a gestão autônoma e descentralizada. 

%% \subsection{git-annex}\label{git-annex}
%% \emph{git-annex}\footnote{\emph{git-annex} é um programa que estende
%%   as funcionalidades do \emph{git} em gerir arquivos de grande tamanho
%% disponível no \url{http://git-annex.branchable.com}.} permite a
%% gestão de arquivos com \emph{git}, sem a necessidade de adicionar os
%% arquivos dentro \emph{git}. Mesmo se pode parecer paradoxal, é útil
%% quando se trabalha com arquivos muito grandes que \emph{git}
%% atualmente não gerencia facilmente por limitações devidas a memoria,
%% tempo ou espaço no disco.

%% Mesmo sem manter o histórico das mudanças do conteúdo do arquivo, ter
%% a possibilidade de gerenciar arquivos com \emph{git}, de movê-los, e
%% exclui-los, numa árvore de pasta versionada, com uso de
%% \emph{branches} e de clones distribuídos, são todos bons motivos para
%% usar \emph{git}. E os arquivos anexos (por isso o nome
%% \emph{git-annex}) podem coexistir no mesmo repositório \emph{git} com
%% os arquivos regularmente versionados. 

%% \emph{git-annex} transforma os arquivos anexos em \emph{link}
%% simbólicos, que são normalmente versionados por \emph{git}. 

%% O conteúdo dos arquivos é mantido por \emph{git-annex} em um acervo
%% chave/valor distribuído que corresponde aos clones de um dado
%% repositório \emph{git}. Praticamente \emph{git-annex} memoriza o
%% conteúdo do arquivo em uma subpasta de \verb|.git/annex/|.

%% A primeira vez que um arquivo é adicionado no \emph{git-annex}, é
%% calculada uma chave, normalmente fazendo um \emph{hash} do seu
%% conteúdo. \emph{git-annex} todavia suporta vários \emph{backend} que
%% podem produzir diferentes tipos de chaves. O arquivo que é adicionado
%% no \emph{git} nada mais é que um \emph{link} simbólico para a chave
%% memorizada no \verb|.git/annex/|. Se o conteúdo do arquivo for
%% modificado, é gerada uma outra chave, e o \emph{link} é alterado. 

%% O conteúdo do arquivo pode ser transferido de um repositório para
%% outro por \emph{git-annex}, que além de manter controle de quem mantem
%% o que, permite criar um mapa das copias disponíveis e impor um número
%% mínimo de cópias. Essas informações são mantidas em um \emph{branch}
%% separado, chamado ``\emph{git-annex}'', e as operações de
%% sincronização, são simplesmente \emph{push} e \emph{pull} entre os
%% vários clones dos repositórios.

%% \emph{git-annex} suporta:
%% \begin{itemize}
%% \item localização das cópias (\emph{location tracking})
%% \item download seletivo dos conteúdos 
%% \item gestão da confiança dos repositórios
%% \item gestão do número minimo de cópias
%% \item vários \emph{backend} para as chaves (SHA\footnote{Secure Hash
%%     Algoritm, (SHA), é um algoritmo usado em sistemas chave/valor onde
%%     as chaves são calculadas através de uma função criptográfica dos
%%     valores.}, WORM\footnote{O algoritmo WORM identifica os arquivos
%%     em base ao nome, dimensão e data de alteração.})
%% \item vários \emph{backend} para os conteúdos/valores
%%   (BUP\footnote{BUP é um sistema para \emph{backup} a alta eficiência
%%     disponível no: \url{https://github.com/apenwarr/bup}.}, rsync,
%%   web, S3\footnote{Amazon Simple Storage Service, (S3) é uma
%%     infraestrutura para a memorização dos dados totalmente redundante,
%%     disponível no: \url{aws.amazon.com/}.})
%% \end{itemize}

%% \section{Gestão de repositórios multiplos}
%% A arquitetura baseadas em repositórios se adapta ao contexto de
%% ``redes federadas'' onde para cada ``Rede'' corresponde um
%% repositório. A solução proposta no Baobáxia inclui já a possibilidade
%% de cadastrar e gerenciar varias redes. A associação rede/repositório
%% permite a escalabilidade da arquitetura e uma gestão diferenciada dos
%% conteúdos. Cada mucua pode se associar a diferentes redes, por
%% exemplo, Rede Mocambos, Estaçoes Digitais, etc.

%% \begin{figure}[htbp]
%%   \centering
%%   \includegraphics[width=\textwidth]{./Fig/Auto_UML_Diagram.pdf}
%%   \rule{35em}{0.5pt}
%%   \caption[Diagrama UML do BBX]{Diagrama UML do BBX}
%%   \label{fig:SchemaServer_ReteMocambos}
%% \end{figure}

%% \subsection{API e rotas}
%% A API REST proposta prevê a gestão de repositórios multiplos em
%% diferentes mucuas, usando o padrão de endereços:
%% $$ /repositorio/mucua/ $$ 

%% Por exemplo, para acessar um ``media'' (uuid
%% aa9f9019-e4f2-4040-bc46-c2e15b66bebc) publicado na ``Rede Mocambos'', na
%% mucua ``Dandara'' o endereço é:
%% $$ /mocambos/dandara/media/aa9f9019-e4f2-4040-bc46-c2e15b66bebc $$

%% ou no caso da ``Estaçao Digital'' na ``Samambaia'' seria:
%% $$ /fbb/samambaia/media/aa9f9019-e4f2-4040-bc46-c2e15b66bebc $$

%% O detalhamento da API e das rotas será definido em capitulo especifico. % REF.

%% \subsection{Autenticação}
%% O Baobáxia pode gerenciar dados de varias redes/organizações e
%% portanto precisa diferenciar os usuários por Rede e Mucua\footnote{A
%%   mucua de alguma forma identifica uma comunidade}. No caso as
%% credenciais dos usuários são mantidas em arquivos versionados pelo git
%% nos repositórios correspondentes.

%% Para manter compatibilidade com outras aplicações do django mantivemos
%% o User padrão do django usando somente o campo username para codificar
%% as informarções necessárias.

%% Um mocambola\footnote{Usuário} é definido por nome, mucua e
%% repositorio no padrão:
%% $$ nome@mucua.repositorio.tld $$ .

%% O django é muito flexivel e suporta diferentes sistemas de
%% autenticação, com a possibilidade de definir seu proprio backend
%% personalizado.

%% O backend de autenticação personalizado, ``FileBackend'' se encontra
%% em \emph{bbx/auth.py}.

%% No BBX, os usuários são serializados em formato \emph{json} e
%% armazenados seguindo o seguinte padrão de localização:
%% $$ /repositorio/mucua/mocambolas/usuario.json $$

%% Por exemplo, no caso do mocambola ``zumbi'' da Casa de Cultura
%% Tainã cuja mucua se chama ``dandara'':
%% $$ zumbi@dandara.mocambos.net $$
%% $$ /mocambos/dandara/mocambolas/zumbi@dandara.mocambos.net $$

%% Notar que o repositório não inclui a extensão de dominio de primeiro
%% nível, no caso do exemplo ``.net'', que permanece no username do
%% mocambola.



