\chapter{NPDD}\label{NPDD}\lhead{\leftmark}
O Núcleo de Pesquisa e Desenvolvimento Digital, NPDD, envolve pessoas
com conhecimento técnicos de varias comunidade e realidades da Rede. O
NPDD é responsável pelo desenvolvimento e manutenção das ferramentas
digitais da Rede, cuidando atualmente dos portais (www.mocambos.net,
wiki.mocambos.net, mapa.mocambos.net, galeria.mocambos.net), das
contas email (@mocambos.net e @mocambos.org) e de criar documentação
de base sobre as ferramentas digitais.

\section{Informações}

\subsection{Endereços e contatos}
Para informações e contatos podem escrever ao seguinte email, nosso
principal canal de comunicação: \\ \url{suporte@mocambos.org}

Existe também uma lista de discussão do NPDD hospedada no riseup.net:
\\ \url{mocambos-npdd@lists.riseup.net}

\begin{tabular}{lll}

\parbox[t]{0.3\textwidth}{
        \textbf{Distrito Federal} \\
        Mercado Sul \\
        QSB 12/13 Loja 7 \\
        Taguatinga/DF \\
}
        &
\parbox[t]{0.3\textwidth}{
        \textbf{São Paulo} \\
        Casa de Cultura Tainã \\
        Rua Inhambú, 645 \\
        Campinas/SP \\
        Telefone: (19) 32282993 \\
}
        &
\parbox[t]{0.35\textwidth}{
        \textbf{Sicilia/Itália} \\
        BOCS \\
        Via Piersanti Mattarella, 8 \\
        Bagheria \\
}

\end{tabular}

\section{Consultorias}
\subsection{Dynamite}
A versão 0.1 do projeto que foca na questão de metadados, conforme
plano de trabalho aprovado, foi desenvolvida com a consultoria da
Associação Cultural Dynamite (nota fiscal 0000999).


\section{Metodologia}
O projeto é conduzido pelo NPDD (Núcleo de Pesquisa e Desenvolvimento
Digital da Rede Mocambos), em sinergia com os demais Núcleos (NCP de
Comunicação e Pedagogia, NFC de Formação Continuada). À equipe fixa de
desenvolvedores se soma a participação de colaboradores e
especialistas em diferentes áreas. Sob a coordenação do NPDD, o
desenvolvimento se da de forma distribuída, utilizando-se ferramentas
colaborativas via internet como git, wiki e irc. A metodologia de
desenvolvimento segue o modelo ``Agile'' que prevê o lançamento
frequente de código funcionante para avaliação contínua por parte dos
usuários finais, permitindo a correção e a melhoria ao longo do
trabalho.

\subsection{Wiki}
A documentação do projeto é disponível no wiki da Rede Mocambos no
endereço: \\ \url{http://wiki.mocambos.net/NPDD/Baobáxia}

\subsection{Codigo}
O codigo do projeto é disponível em licença GPLv3 no Github no
endereço: \\ \url{http://github.com/RedeMocambos/baobaxia}

\subsection{Necessidades/Issues}
As necessidades do projeto são registradas no sistema de issues do
github no endereço:
\\ \url{http://github.com/RedeMocambos/baobaxia/issues}

