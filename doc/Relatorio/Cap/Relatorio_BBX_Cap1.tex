\chapter{NPDD}\label{NPDD}\lhead{\leftmark}

O Núcleo de Pesquisa e Desenvolvimento Digital, NPDD, envolve pessoas
com conhecimento técnicos de varias comunidade e realidades da Rede. O
NPDD é responsável pelo desenvolvimento e manutenção das ferramentas
digitais da Rede, cuidando atualmente dos portais (www.mocambos.net,
wiki.mocambos.net, mapa.mocambos.net, galeria.mocambos.net), das
contas email (@mocambos.net e @mocambos.org) e de criar documentação
de base sobre as ferramentas digitais.

\section{Informações}

\subsection{Endereços}

\begin{minipage}[t]{0.4\textwidth}
        \textbf{Distrito Federal} \\
        Mercado Sul \\
        QSB 12/13 Loja 7 \\
        Taguatinga/DF \\
\end{minipage}

\begin{minipage}[t]{0.4\textwidth}
        \textbf{São Paulo} \\
        Casa de Cultura Tainã \\
        Rua Inhambú, 645 \\
        Campinas/SP \\
        Telefone: (19) 32282993 \\
\end{minipage}


\section{Metodologia}
O projeto será conduzido pelo NPDD (Núcleo de Pesquisa e
Desenvolvimento Digital da Rede Mocambos), em sinergia com os demais
Núcleos (NCP de Comunicação e Pedagogia, NFC de Formação
Continuada). À equipe fixa de desenvolvedores se somará a participação
de colaboradores, especialistas em diferentes áreas. Sob a coordenação
do NPDD, o desenvolvimento se dará de forma distribuída, utilizando-se
ferramentas colaborativas via internet como git, wiki e irc. A
metodologia de desenvolvimento seguirá o modelo ``Agile'' que prevê o
lançamento frequente de código funcionante para avaliação contínua por
parte dos usuários finais, permitindo a correção e a melhoria ao longo
do trabalho.

\subsection{Wiki}
A documentação do projeto é disponível no wiki da Rede Mocambos no
endereço: \\
\url{http://wiki.mocambos.net/NPDD/Baobáxia}

\subsection{Codigo}
O codigo do projeto é disponível em licença GPLv3 no Github no
endereço: \\
\url{http://github.com/RedeMocambos/baobaxia}

\subsection{Necessidades/Issues}
As necessidades do projeto são registradas no sistema de issues do
github no endereço: \\
\url{http://github.com/RedeMocambos/baobaxia/issues}

