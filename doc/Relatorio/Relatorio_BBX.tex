%% ----------------------------------------------------------------
%% Thesis.tex -- MAIN FILE (the one that you compile with LaTeX)
%% ---------------------------------------------------------------- 

% Set up the document
\documentclass[a4paper, 11pt, oneside]{Relatorio}  % Use the "Relatorio" style, based on the ECS Thesis style by Steve Gunn
\graphicspath{{Figure/}}  % Location of the graphics files (set up for graphics to be in PDF format)


% Include any extra LaTeX packages required
\usepackage[utf8]{inputenc}
\usepackage[brazilian]{babel}
\usepackage[T1]{fontenc}
\usepackage{ae,aecompl}
\usepackage[square, numbers, comma, sort&compress]{natbib}  % Use the "Natbib" style for the references in the Bibliography
\usepackage{verbatim}  % Needed for the "comment" environment to make LaTeX comments
%\usepackage{vector}  % Allows "\bvec{}" and "\buvec{}" for "blackboard" style bold vectors in maths
%\usepackage{listings}
\usepackage{listingsutf8}
\lstset{frame=tb,
  language=Python,
  aboveskip=3mm,
  belowskip=3mm,
  showstringspaces=false,
  columns=flexible,
  basicstyle={\small\ttfamily},
  numbers=none,
  numberstyle=\tiny\color{gray},
  keywordstyle=\color{dkblue},
  commentstyle=\color{dkgreen},
  stringstyle=\color{mauve},
  breaklines=true,
  breakatwhitespace=true,
  tabsize=4,
  lineskip={-1.5pt},
  morekeywords={models, lambda, forms, =}
  inputencoding=utf8,
  extendedchars=\true
}

\lstnewenvironment{code}[1][]%
  {\minipage{\linewidth} 
   \lstset{basicstyle=\ttfamily\footnotesize,frame=tb,#1}}
  {\endminipage}


%%%% Color definitions
\usepackage{color}
\definecolor{dkgreen}{rgb}{0,0.6,0}
\definecolor{dkblue}{rgb}{0,0,0.5}
\definecolor{gray}{rgb}{0.5,0.5,0.5}
\definecolor{mauve}{rgb}{0.6,0,0.82}
\definecolor{coolblack}{rgb}{0.0, 0.18, 0.39}
\hypersetup{urlcolor=coolblack, colorlinks=true}  % Colours hyperlinks in blue, but this can be distracting if there are many links.
%\hypersetup{urlcolor=black, colorlinks=false}  

%% To show a border around all figures
%\usepackage{float}
%\floatstyle{boxed} 
%\restylefloat{figure}

%% ----------------------------------------------------------------
\begin{document}
\frontmatter	  % Begin Roman style (i, ii, iii, iv...) page numbering

% Set up the Title Page
\title  {Baobáxia}

\authors  {\texorpdfstring
            {\href{mailto:vince@mocambos.net}{Vincenzo Tozzi}} - 
            {\href{mailto:fernao@riseup.net}{Fernão Lopes}}
            }
\addresses  {\taina\\\npdd\\\redemocambos}  % Do not change this here, instead these must be set in the "Thesis.cls" file, please look through it instead
\date       {\today}
\subject    {Relatório de execução do projeto}
\keywords   {Federated Network, Tecnological Autonomy, Free Software, Rede Mocambos, Casa de Cultura Taina, Quilombo}

\maketitle
%% ----------------------------------------------------------------

\setstretch{1.3}  % It is better to have smaller font and larger line spacing than the other way round

% Define the page headers using the FancyHdr package and set up for one-sided printing
\fancyhead{}  % Clears all page headers and footers
\rhead{\thepage}  % Sets the right side header to show the page number
\lhead{}  % Clears the left side page header

\pagestyle{fancy}  % Finally, use the "fancy" page style to implement the FancyHdr headers

% %% ----------------------------------------------------------------
% % Declaration Page required for the Thesis, your institution may give you a different text to place here
% \Declaration{

% \addtocontents{toc}{\vspace{1em}}  % Add a gap in the Contents, for aesthetics

% Io, Vincenzo Tozzi, dichiaro che la presente tesi, "Reti federate eventualmente connesse" e il lavoro presentato in essa sono di mia paternità. Dichiaro che:

% \begin{itemize} 
% \item[\tiny{$\blacksquare$}] Questo lavoro è stato totalmente o per la maggior parte svolto come laureando per la Laurea in Informatica di questa Università.
 
% \item[\tiny{$\blacksquare$}] Ho esplicitamente dichiarato nel testo se qualche parte di questa tesi è stata precedentemente pubblicata in altri lavori da questa o altre Università o istituzioni.
 
% \item[\tiny{$\blacksquare$}] Ho attribuito la paternità ai lavori pubblicati da altri e da me consultati.
 
% \item[\tiny{$\blacksquare$}] Ho sempre citato la fonte di opere altrui. Ad eccezione di tali citazioni, questa tesi è di mia paternità.
 
% \item[\tiny{$\blacksquare$}] Ho ringranziato tutte le principali fonti di supporto.
 
% \item[\tiny{$\blacksquare$}] Ho esplicitamente dichiarato del testo, in parti sviluppate assieme ad altri, qual'è il loro e il mio contributo.
% \\
% \end{itemize}
 
 
% Firmato:\\
% \rule[1em]{25em}{0.5pt}  % This prints a line for the signature
 
% Data:\\
% \rule[1em]{25em}{0.5pt}  % This prints a line to write the date
% }
% \clearpage  % Declaration ended, now start a new page

%% ----------------------------------------------------------------
% The "Funny Quote Page"
\pagestyle{empty}  % No headers or footers for the following pages

\null\vfill
% Now comes the "Funny Quote", written in italics
\textit{``Vamos fazer um mundo mais do nosso jeito...''}

\begin{flushright}
Zumbi dos Palmares
\end{flushright}

\vfill\vfill\vfill\vfill\vfill\vfill\null
\clearpage  % Funny Quote page ended, start a new page
%% ----------------------------------------------------------------

% The Abstract Page
\addtotoc{Resumo}  % Add the "Abstract" page entry to the Contents
\abstract{

\addtocontents{toc}{\vspace{1em}}  % Add a gap in the Contents, for aesthetics

O projeto Baobáxia trata da concepção, desenvolvimento e implementação
de uma arquitetura distribuída, voltada para a integração de redes
locais mesmo em localidades nas quais a conexão à internet seja
instável, lenta ou intermitente. Parte-se da experiência acumulada
pela Rede Mocambos, que trabalha com a integração de duzentas
comunidades em todas as regiões do país através da apropriação de
tecnologias, para identificar pontos críticos em que a precariedade do
acesso à internet se torna um impeditivo para a efetiva comunicação
entre essas comunidades. O projeto parte do pressuposto de que não
basta usar tecnologias de informação já existentes - precisamos moldar
o próprio desenvolvimento dessas tecnologias para que atendam às
demandas da sociedade. E adota como princípio básico e metodologia de
trabalho os fundamentos do software livre - tanto na gestão das
equipes de trabalho quanto nas soluções tecnológicas que utilizará.

}

\clearpage  % Abstract ended, start a new page
%% ----------------------------------------------------------------

\setstretch{1.3}  % Reset the line-spacing to 1.3 for body text (if it has changed)

% The Acknowledgements page, for thanking everyone
%% \acknowledgements{
%% \addtocontents{toc}{\vspace{1em}}  % Add a gap in the Contents, for aesthetics

%% Um agradecimento especial para as nossas familias\ldots

%% }
\clearpage  % End of the Acknowledgements
%% ----------------------------------------------------------------

\pagestyle{fancy}  %The page style headers have been "empty" all this time, now use the "fancy" headers as defined before to bring them back


%% ----------------------------------------------------------------
\lhead{\emph{Contents}}  % Set the left side page header to "Contents"
\tableofcontents  % Write out the Table of Contents

%% ----------------------------------------------------------------
%\lhead{\emph{List of Figures}}  % Set the left side page header to "List if Figures"
%\listoffigures  % Write out the List of Figures

%% ----------------------------------------------------------------
%\lhead{\emph{List of Tables}}  % Set the left side page header to "List of Tables"
%\listoftables  % Write out the List of Tables

%% ----------------------------------------------------------------
\setstretch{1.5}  % Set the line spacing to 1.5, this makes the following tables easier to read
\clearpage  % Start a new page
\lhead{\emph{Acrónimos}}  % Set the left side page header to "Abbreviations"
\listofsymbols{ll}  % Include a list of Abbreviations (a table of two columns)
{
% \textbf{Acronym} & \textbf{W}hat (it) \textbf{S}tands \textbf{F}or \\
  \textbf{RM} & \textbf{R}ede \textbf{M}ocambos\\ \textbf{SP} &
  \textbf{S}ervice \textbf{P}rovider\\ \textbf{IdP} &
  \textbf{Id}entity \textbf{P}rovider\\ \textbf{API} &
  \textbf{A}pplication \textbf{P}rogramming
  \textbf{I}nterface\\ \textbf{RFC} & \textbf{R}equest \textbf{F}or
  \textbf{C}omments\\ \textbf{JSON} & \textbf{J}ava\textbf{S}cript
  \textbf{O}bject \textbf{N}otation\\ \textbf{P2P} & \textbf{P}eer To
  \textbf{P}eer\\ \textbf{LDAP} & \textbf{L}ightweight
  \textbf{D}irectory \textbf{A}ccess \textbf{P}rotocol\\ \textbf{YAML}
  & \textbf{Y}et \textbf{A}nother \textbf{M}arkup
  \textbf{L}anguage\\ \textbf{XMPP} & e\textbf{X}tensible
  \textbf{M}essaging and \textbf{P}resence
  \textbf{P}rotocol\\ \textbf{SSO} & \textbf{S}ingle \textbf{S}ign
  \textbf{O}n\\ \textbf{VSAT} & \textbf{V}ery \textbf{S}mall
  \textbf{A}perture \textbf{T}erminal\\ \textbf{DRY} & \textbf{D}on't
  \textbf{R}epeat \textbf{Y}ourself\\ \textbf{MVC} & \textbf{M}odel
  \textbf{V}iew \textbf{C}ontroller\\ \textbf{ORM} & \textbf{O}bject
  \textbf{R}elational \textbf{M}apper\\ \textbf{NPDD} &
  \textbf{N}úcleo de \textbf{P}esquisa e \textbf{D}esenvolvimento
  \textbf{D}igital\\ \textbf{NFC} & \textbf{N}úcleo de
  \textbf{F}ormação \textbf{C}ontinuada\\ \textbf{NCP} &
  \textbf{N}úcleo de \textbf{C}omunicação e
  \textbf{P}edagogia\\ \textbf{GESAC} & \textbf{G}overno
  \textbf{E}letrônico \textbf{S}erviço de \textbf{A}tendimento ao
  \textbf{C}idadão\\ }

%% ----------------------------------------------------------------
%\clearpage  % Start a new page
%\lhead{\emph{Physical Constants}}  % Set the left side page header to "Physical Constants"
%\listofconstants{lrcl}  % Include a list of Physical Constants (a four column table)
%{
%% Constant Name & Symbol & = & Constant Value (with units) \\
%Speed of Light & $c$ & $=$ & $2.997\ 924\ 58\times10^{8}\ \mbox{ms}^{-\mbox{s}}$ (exact)\\
%
%}

%% ----------------------------------------------------------------
%\clearpage  %Start a new page
%\lhead{\emph{Simboli}}  % Set the left side page header to "Symbols"
%\listofnomenclature{lll}  % Include a list of Symbols (a three column table)
%{
%% symbol & name & unit \\
%$a$ & distance & m \\
%$P$ & power & W (Js$^{-1}$) \\
%& & \\ % Gap to separate the Roman symbols from the Greek
%$\omega$ & angular frequency & rads$^{-1}$ \\
%}
%% ----------------------------------------------------------------
% End of the pre-able, contents and lists of things
% Begin the Dedication page

%% \setstretch{1.3}  % Return the line spacing back to 1.3

%% \pagestyle{empty}  % Page style needs to be empty for this page
%% \dedicatory{Para nossa Mãe\ldots}

%% \addtocontents{toc}{\vspace{2em}}  % Add a gap in the Contents, for aesthetics


%% ----------------------------------------------------------------
\mainmatter	  % Begin normal, numeric (1,2,3...) page numbering
\pagestyle{fancy}  % Return the page headers back to the "fancy" style

% Include the chapters of the thesis, as separate files
% Just uncomment the lines as you write the chapters

\input{./Relatorio_BBX_Conteudos.tex} % Architettura

%\input{./Capitoli/Capitolo1} % Introduzione

%\input{./Capitoli/Capitolo2} % Reti Federate  

\clearpage  % To start a new page

%% ----------------------------------------------------------------
% The "Funny Quote Page"
\pagestyle{empty}  % No headers or footers for the following pages

\null\vfill
% Now comes the "Funny Quote", written in italics
\textit{``A força da rede esta nos nós''}

\begin{flushright}
TC
\end{flushright}

\vfill\vfill\vfill\vfill\vfill\vfill\null
\clearpage  % Funny Quote page ended, start a new page
%% ----------------------------------------------------------------

\pagestyle{fancy}  % Finally, use the "fancy" page style to implement
                   % the FancyHdr headers

%\input{./Capitoli/Capitolo4} % Prototipo

%\input{./Capitoli/Capitolo5} % Conclusioni

%\input{./Chapters/Chapter6} % Results and Discussion

%\input{./Chapters/Chapter7} % Conclusion

%% ----------------------------------------------------------------
% Now begin the Appendices, including them as separate files

\addtocontents{toc}{\vspace{2em}} % Add a gap in the Contents, for aesthetics

\appendix % Cue to tell LaTeX that the following 'chapters' are Appendices

% Appendice A                                                                                                                                                                   
\chapter{Listagem do código}
\label{ApendiceA}
\lhead{Apêndice A \emph{Listagem do código}}

\section{gitannex}

\lstinputlisting[basicstyle={\scriptsize\ttfamily}]{../../tests/django-backbone_0/gitannex/README}

\subsection{gitannex/admin.py}
\lstinputlisting[basicstyle={\scriptsize\ttfamily}]{../../tests/django-backbone_0/gitannex/admin.py}

\subsection{gitannex/models.py}
\lstinputlisting[basicstyle={\scriptsize\ttfamily}]{../../tests/django-backbone_0/gitannex/models.py}

\subsection{gitannex/signals.py}
\lstinputlisting[basicstyle={\scriptsize\ttfamily}]{../../tests/django-backbone_0/gitannex/signals.py}

\subsection{gitannex/management/commands/run scheduled jobs.py}
\lstinputlisting[basicstyle={\scriptsize\ttfamily}]{../../tests/django-backbone_0/gitannex/management/commands/run_scheduled_jobs.py}


%% \section{mmedia}

%% \lstinputlisting[basicstyle={\scriptsize\ttfamily}]{src/mmedia/README}

%% \subsection{mmedia/admin.py}
%% \lstinputlisting[basicstyle={\scriptsize\ttfamily}]{src/mmedia/admin.py}

%% \subsection{mmedia/models.py}
%% \lstinputlisting[basicstyle={\scriptsize\ttfamily}]{src/mmedia/models.py}

%% \subsection{mmedia/signals.py}
%% \lstinputlisting[basicstyle={\scriptsize\ttfamily}]{src/mmedia/signals.py}

%% \subsection{mmedia/forms.py}
%% \lstinputlisting[basicstyle={\scriptsize\ttfamily}]{src/mmedia/forms.py}

%% \subsection{mmedia/management/commands/create objects from files.py}
%% \lstinputlisting[basicstyle={\scriptsize\ttfamily}]{src/mmedia/management/commands/create_objects_from_files.py}
	% Appendice codice sorgente

%\input{./Appendices/AppendixB} % Appendix Title

%\input{./Appendices/AppendixC} % Appendix Title

\addtocontents{toc}{\vspace{2em}}  % Add a gap in the Contents, for aesthetics
\backmatter
\nocite{*}
%% ----------------------------------------------------------------
\label{Bibliography}
\lhead{\emph{Bibliografia}}  % Change the left side page header to "Bibliography"
\bibliographystyle{unsrtnat}  % Use the "unsrtnat" BibTeX style for formatting the Bibliography
\bibliography{Bibliography}  % The references (bibliography) information are stored in the file named "Bibliography.bib"

\end{document}  % The End
%% ----------------------------------------------------------------
