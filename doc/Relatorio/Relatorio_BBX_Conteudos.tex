\label{NPDD}
\lhead{NPDD \emph{Núcleo de Pesquisa e Desenvolvimento Digital}}

\chapter{NPDD}

O Núcleo de Pesquisa e Desenvolvimento Digital, NPDD, envolve pessoas
com conhecimento técnicos de varias comunidade e realidades da Rede. O
NPDD é responsável pelo desenvolvimento e manutenção das ferramentas
digitais da Rede, cuidando atualmente dos portais (www.mocambos.net,
wiki.mocambos.net, mapa.mocambos.net, galeria.mocambos.net), das
contas email (@mocambos.net e @mocambos.org) e de criar documentação
de base sobre as ferramentas digitais.

\section{Informações}

\subsection{Endereços}

\begin{minipage}[t]{0.45\textwidth}
        \textbf{Distrito Federal} \\
        Mercado Sul \\
        QSB 12/13 Loja 7 \\
        Taguatinga/DF \\
\end{minipage}

\begin{minipage}[t]{0.45\textwidth}
        \textbf{São Paulo} \\
        Casa de Cultura Tainã \\
        Rua Inhambú, 645 \\
        Campinas/SP \\
        Telefone: (19) 32282993 \\
\end{minipage}


\section{Metodologia}
O projeto será conduzido pelo NPDD (Núcleo de Pesquisa e
Desenvolvimento Digital da Rede Mocambos), em sinergia com os demais
Núcleos (NCP de Comunicação e Pedagogia, NFC de Formação
Continuada). À equipe fixa de desenvolvedores se somará a participação
de colaboradores, especialistas em diferentes áreas. Sob a coordenação
do NPDD, o desenvolvimento se dará de forma distribuída, utilizando-se
ferramentas colaborativas via internet como git, wiki e irc. A
metodologia de desenvolvimento seguirá o modelo ``Agile'' que prevê o
lançamento frequente de código funcionante para avaliação contínua por
parte dos usuários finais, permitindo a correção e a melhoria ao longo
do trabalho.

\subsection{Wiki}
A documentação do projeto é disponível no wiki da Rede Mocambos no
endereço: \\
\url{http://wiki.mocambos.net/NPDD/Baobáxia}

\subsection{Codigo}
O codigo do projeto é disponível em licença GPLv3 no Github no
endereço: \\
\url{http://github.com/RedeMocambos/baobaxia}

\subsection{Necessidades/Issues}
As necessidades do projeto são registradas no sistema de issues do
github no endereço: \\
\url{http://github.com/RedeMocambos/baobaxia/issues}



\label{Metadados}
\lhead{Metadados \emph{Metadados}}

\chapter{Metadados}

\section{Descrição geral da pesquisa de Metadados e políticas de metadados}
A pesquisa sobre metadados para o sistema Baobáxia compreendeu as
definições diretamente relacionadas aos metadados das mídias e também
a participação na modelagem geral dos objetos do sistema, assim como a
definição de formatos para troca de dados. Por metadados entendemos
toda a informação a respeito de si, seja para a mídia, seja para as
Mucuas como para o sistema. Deste modo, num entendimento amplo de
metadados, consideramos que qualquer dado descritivo a respeito do
sistema como um todo deve ser entendido como um metadado, e logo deve
ser armazenado em arquivos de definição.

Essa diretiva partiu de uma orientação em conjunto com o
desenvolvimento do núcleo do software Baobáxia
(https://github.com/RedeMocambos/baobaxia). Deste modo, tudo que
envolve a descrição de elementos do sistema pode ser considerado um
metadado. Os metadados, embora tenham como objetivo a estabilidade da
informação, devem contemplar a possibilidade de alterações nas suas
definições, bem como a expansão futura.

De início, definem-se dois tipos principais de metadado:
\begin{itemize}
\item metadado do sistema / políticas (arquivos de configuração,
  regras de funcionamento)
\item metadado do acervo/das mídias
\end{itemize}

Não coube à pesquisa em metadados a definição das políticas, mas a
orientação de como estas deveriam ser armazenadas e estruturadas. As
políticas de sistema dizem respeito a definições gerais sobre a
configuração do software e de critérios estabelecidos politicamente,
isto é, atendendo às demandas da comunidade usuária e gestora do
sistema Baobáxia. Entretanto, há uma escolha que entende a importância
da distinção entre o sistema e suas configurações, estas que devem ser
implementadas de acordo com as necessidades locais da Mucua ou do
repositório (baobáxia / Rede Mocambos).

A partir da definição de políticas, ocorre a tomada de decisões pelo
sistema, respeitando aos critérios estabelecidos. Essa definição
possui normas técnicas próprias, mas é abrangente o bastante para
permitir que novas políticas sejam incorporadas ao sistema conforme
surjam novas necessidades.

\section{Mídias e Mucuas}
Os metadados relacionados a mídias e mucuas constituem o núcleo do
sistema. Em tratando-se de um sistema de acervo distribuído, os nós
(Mucuas) e arquivos contidos nesses nós (Mídias) são os principais
elementos do sistema do ponto de vista dos conteúdos.

As mídias sempre estão associadas a mucuas originadoras (as quais originaram a publicação), e dividem-se por tipos:
\begin{itemize}
\item vídeos
\item imagens (fotografias, desenhos etc)
\item áudios (músicas, entrevistas, programas de rádio etc)
\item arquivos (documentos, textos e demais arquivos)
\end{itemize}

Cada tipo de mídia possui especificidades do ponto de vista de seus
metadados, ao que buscou-se definir o que há de comum entre todas. Os
metadados foram definidos preliminarmente como o mínimo essencial
possível, sendo que qualificadores adicionais devem ser estendidos ou
como tags ou como metadados específicos do tipo de arquivo. Assim,
chegou-se aos seguintes campos, cuja definição encontra-se nos
arquivos de políticas (ver adiante):
\begin{itemize}
\item data (AAAA-MM-DD)
\item uuid (identificação única do arquivo)
\item title (título, texto não único)
\item comment (comentário, texto aberto)
\item author (mocambola, associado a quem publicou o arquivo)
\item type (tipos das mídias [vídeo|imagens|áudios|arquivos])
\item format (formato do arquivo, definido em arquivo de políticas)
\item origin (mucua, o servidor a partir do qual arquivo foi
  publicado)
\item repository (referência ao repositório git annex a que está vinculado o arquivo)
\item tags (etiquetas, múltiplas; somente descritoras ou também
  funcionais)
\end{itemize}

Além da mídia, as mucuas (nós do acervo, servidores locais) possuem
também metadados específicos, bem como associados. Os metadados
específicos dizem respeito a informações ligadas diretamente ao
servidor; as associadas, ao conteúdo hospedado no servidor. Dessa
forma, temos:
\begin{itemize}
\item note (texto geral sobre a mucua)
\item description (nome da mucua)
\item uuid (identificador único da mucua)
\end{itemize}


Quanto ao metadado associado, diz respeito a todas as etiquetas de
conteúdos que estejam hospedados na mucua. Está previsto como um
metadado descritivo sobre a mucua um relatório agrupando todas as
etiquetas dos arquivos hospedados, o que permite aos mocambolas
(usuári@s) ter uma ideia geral sobre as características do acervo de
determinada Mucua.

\section{Armazenamento do metadado: arquivos, padrões adotados e intercâmbio}
Conforma já assinalado, metadados e políticas são armazenados em
arquivos descritores. Para tal foi feita uma pesquisa a respeito dos
formatos a serem adotados para armazenamento destes dados. Teriam que
atender os seguintes requisitos:
\begin{description}
\item[A] armazenamento em suporte aberto, não proprietário
\item[B] ótima capacidade de intercâmbio entre linguagens
\item[C] amplo desenvolvimento de bibliotecas de código em distintas linguagens de programação
\item[D] capacidade para armazenamento de objetos complexos
\item[E] bom desempenho
\item[F] tamanho diminuto
\item[G] adequação a plataformas RESTful e serviços de dados (web services)
\end{description}

Inicialmente, levantou-se três possibilidades, todas elas atendendo os itens A e B:
\begin{description}
\item[XML] eXpansible Markup Language
\item[YAML] Yet Another Markup Languge
\item[JSON] JavaScript Object Notation
\end{description}

O primeiro formato, XML, tem a seu favor o fato de ser largamente
adotado para intercâmbio de dados e para descrição de informações já
há muitos anos (C). Conta no entanto com algumas desvantagens
sobretudo do ponto de vista do desempenho (E), além de gerar uma não
desprezível quantidade de informações extra especialmente ao lidar com
o armazenamento de objetos complexos, o que tem impactos severos ao se
pensar numa rede de acervos com conectividade baixa ou nula (ponto
F). Além disso, conta com possíveis problemas no armazenamento de
objetos complexos ao redundar numa estrutura de dados pesada (D).

O segundo formato, YAML, apresenta-se como um possível sucessor do
XML, tendo se inspirado neste. Tem como resultado arquivos sintéticos
e de tamanho diminuto (E e F). Possui ótima capacidade de
armazenamento de objetos complexos (D). Apesar de contar com mais
funcionalidades que o JSON - como a possibilidade de estabelecer
relacionamentos funcionais lógicos internos, conta ainda com menor
desenvolvimento e compatibilidade (C e G), ainda que seu futuro seja
promissor.

O terceiro formato, JSON, é considerado uma forma de reduzir o
overhead computacional do XML (E), sendo uma excelente alternativa
para armazenamento de objetos complexos (D), sendo sintético e de
tamanho diminuto (F). Conta com grande desenvolvimento em uma série de
softwares (C) pode-se dizer que atualmente é o novo padrão para
intercâmbio de dados, sendo base para numerosas tecnologias baseadas
em plataformas RESTful (G)

Desse modo, foi escolhido o formato JSON como padrão para intercâmbio
de dados, definição de políticas e metadados.

\section{Politicas}

\subsection{Media}

Definição dos metadados de mídia:
\begin{description}
  \item[formats] definem-se os tipos de formatos aceitos
  \item[priority] tipos prioritários aceitos pelo sistema
  \item[metadata] definição dos campos com tipo de dados aceito. Pode
    receber uma expressão regular (ex.: \verb| /^\w_-$/ |)
\end{description}

\lstinputlisting[basicstyle={\scriptsize\ttfamily}]{../../policies/media.json}

\subsection{Mucua}

\lstinputlisting[basicstyle={\scriptsize\ttfamily}]{../../policies/mucua.json}


